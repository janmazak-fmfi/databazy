% use xetex
\documentclass[10pt, a4paper]{article}

\usepackage{ifxetex}
\ifxetex
    \usepackage{fontspec}
    \usepackage{polyglossia}
    \setmainlanguage{slovak}
    \setotherlanguage{english}
    \usepackage{fontspec}
    \setmonofont[Scale=.8]{FreeMono}
\else
    % some unicode chars rely on xetex
    % remove them first to use latex
    \errmessage{Use xetex!}

    \usepackage[T1]{fontenc}
    \usepackage[utf8]{inputenc}
    \usepackage[activeacute, slovak]{babel}
    \usepackage{lmodern}
\fi

% rozmery stranky
\addtolength{\voffset}{-3cm}
\addtolength{\hoffset}{-2.5cm}
\addtolength{\textwidth}{5 cm}
\addtolength{\textheight}{5 cm}
\parindent = 0 pt
\addtolength{\parskip}{0.5\baselineskip}

\usepackage{csquotes}
\usepackage{url}
\usepackage{xcolor}
\usepackage[makeroom]{cancel}


% AMS-TeX
\usepackage{amsmath}
\usepackage{amsfonts}
\usepackage{amssymb}
\usepackage{multicol}
\usepackage{amsthm}

\makeatletter
\def\thm@space@setup{%
  \advance\thm@preskip by 10 mm
  \thm@postskip=\thm@preskip
}
\makeatother

\theoremstyle{definition}
\newtheorem{problem}{Úloha}[section]

\def\beginwrong{\color{red!85!black}}
\def\endwrong{\color{black}}

\def\hr{
    \bigskip
    \hrule
    \bigskip
}

% database-related stuff
\DeclareMathOperator{\join}{\bowtie}
\DeclareMathOperator{\antijoin}{\rhd}

\DeclareMathOperator{\COUNT}{\textrm{COUNT}}
\DeclareMathOperator{\SUM}{\textrm{SUM}}
\DeclareMathOperator{\MAX}{\textrm{MAX}}


\DeclareMathOperator{\osoba}{osoba}
\DeclareMathOperator{\firma}{firma}
\DeclareMathOperator{\vlastni}{vlastni}
\DeclareMathOperator{\ponuka}{ponuka}
\DeclareMathOperator{\chce}{chce}
\DeclareMathOperator{\lubi}{lubi}
\DeclareMathOperator{\capuje}{capuje}
\DeclareMathOperator{\navstivil}{navstivil}
\DeclareMathOperator{\vypil}{vypil}
\DeclareMathOperator{\answer}{answer}



\begin{document}

\pagestyle{empty}

%%%%%%%%%%%%%%%%%%%%%%%%%%%%%%%%%%%%%%%%%%%%%%%%%%%%%%%%%%%%%%%%%%%%%%%%%%%%%%%%%%%%%%%%%%%%%%%%%%%%%%%%%%%%%%%%%%%%%%%%%%%%%%%%%%%%%
v0.8.1

POZOR --- materiál je nekompletný a môže obsahovať rôzne chyby a nezmysly.

(Budem vďačný za návrhy na zlepšenia: {\tt mazak@dcs.fmph.uniba.sk}.)

%%%%%%%%%%%%%%%%%%%%%%%%%%%%%%%%%%%%%%%%%%%%%%%%%%%%%%%%%%%%%%%%%%%%%%%%%%%%%%%%%%%%%%%%%%%%%%%%%%%%%%%%%%%%%%%%%%%%%%%%%%%%%%%%%%%%%
\section{Dotazovacie jazyky}

\begin{itemize}
\item relačný kalkul
\item datalog
\item SQL
\item relačná algebra (na rozdiel od predošlých aj špecifikuje postup výpočtu)
\end{itemize}

Pri premýšľaní nad dotazom (alebo jeho časťou) môžete použiť ľubovoľný z jazykov a potom výsledok prepísať do iného jazyka. Toto sa asi najľahšie realizuje pri prepise z datalogu do SQL alebo relačnej algebry. SQL je jazyk pomerne ťažkopádny a má niekoľko nevýhod (pozri napr. \url{https://web.archive.org/web/20110305230025/http://www.cs.duke.edu/courses/spring03/cps216/papers/date-1983.pdf}), ktoré vznikli tým, že nebol navrhnutý \enquote{profesionálne}, ale skôr narýchlo ad-hoc a potom sa presadil na trhu napriek existencii technicky lepších alternatív.

Nebojte sa jazyky kombinovať. Jazyk je v princípe nástroj, a preto sa netreba báť použiť kombináciu nástrojov na riešenie jedného problému. Napríklad na vyjadrenie komplikovanejšej štruktúry kvantifikátorov datalog, ktorý neskôr mechanicky prepíšeme do SQL, a agregáciu nad tými kvantifikovanými dátami zapíšeme rovno v SQL. (Výhodou datalogu oproti SQL pri kvantifikátoroch je ďaleko stručnejší zápis a hlavne sa nestratíme v množstve podmienok za \verb|WHERE|, ktoré stotožňujú výskyty jednej premennej --- v datalogu takéto niečo máme zadarmo, stačí použiť rovnaké písmeno.)

Voľba jazyka, v ktorom premýšľame nad riešením problému, je veľmi dôležitá. Skúste chvíľu reprezentovať čísla pomocou rímskych číslic (vrátane výpočtov typu násobenie) a rýchlo pochopíte jeden z dôvodov, prečo to Rimania v matematike ďaleko nedotiahli. O nič lepšia nie je reprezentácia čísel pomocou kmeňových zlomkov ako v Egypte. Na druhej strane Arabi s nulou a pozičnou sústavou spravili citeľný pokrok. Alternatívne, predstavte si, ako by ste písali webserver alebo databázové dotazy v asembleri.

Riešenia sú naschvál zapisované s rôznym spôsobom formátovania: vyberte si taký, čo sa vám pozdáva (najmä čo sa týka SQL).
Snažil som sa nepoužiť ťažko čitateľné spôsoby; tie uvedené by mali byť aspoň v princípe OK, aj keď niektoré z nich sa mi pozdávajú viac ako iné.

Pri práci v relačnej algebre či SQL môžete využiť online prostredie RelaX.
Linky nižšie už obsahujú aj databázovú schému pre databázu pijanov (rozdiel je len v názvoch stĺpcov).

\url{https://dbis-uibk.github.io/relax/calc/gist/51930198d605017448b37c128684e555}

\url{https://dbis-uibk.github.io/relax/calc/gist/5cbffeac631d5fab2f1885e5e922841e}


%%%%%%%%%%%%%%%%%%%%%%%%%%%%%%%%%%%%%%%%%%%%%%%%%%%%%%%%%%%%%%%%%%%%%%%%%%%%%%%%%%%%%%%%%%%%%%%%%%%%%%%%%%%%%%%%%%%%%%%%%%%%%%%%%%%%%
\section{Úvodné príklady (relačný kalkul, datalog)}

$EDB = \{\osoba(O), \firma(F), \vlastni(Kto, Co)\}$

V tejto časti uvádzame len správne riešenia úloh, nie príklady nesprávnych riešení.

\begin{problem}
{\bf Firmy vlastnené jedinou osobou a ničím iným.}
\begin{align*}
& \{F \mid \firma(F)\land \exists O (\osoba(O)\land \vlastni(O, F) \land \forall X (\vlastni(X, F) \implies X = O))\}\\
& \{F \mid \firma(F)\land \exists O (\osoba(O)\land \vlastni(O, F) \land \lnot \exists X (\vlastni(X, F) \land X \neq O))\}
\end{align*}
\begin{align*}
\answer(F) & \leftarrow \vlastni(O, F), \osoba(O), \neg \operatorname{inyVlastnik}(F, O).\\
\operatorname{inyVlastnik}(F, O) & \leftarrow \vlastni(O, F), \vlastni(O_2, F), O_2\neq O.
\end{align*}
\begin{align*}
\answer(F) & \leftarrow \vlastni(O, F), \osoba(O), \neg \operatorname{dvajaVlastnici}(F).\\
\operatorname{dvajaVlastnici}(F) & \leftarrow \vlastni(O_1, F), \vlastni(O_2, F), O_1\neq O_2.
\end{align*}
\end{problem}

\begin{problem}
{\bf Koneční užívatelia výhod, t.\,j. osoby, ktoré vlastnia firmu, a nie sú vlastnené inou osobou ani firmou. (Pozor: v databáze zatiaľ nie je žiadna kontrola integrity, čiže môžu byť ako vlastníci uvedené entity, ktoré nie sú osobami ani firmami.)}
\begin{align*}
\{O \mid \osoba(O) \land \exists F (\firma(F)\land \vlastni(O, F))\land\lnot \exists X (\vlastni(X, O) \land (\osoba(X) \lor \firma(X)))\}
\end{align*}
\begin{align*}
\answer(O) & \leftarrow \osoba(O), \firma(F), \vlastni(O, F), \neg \operatorname{owned}(O).\\
\operatorname{owned}(O) & \leftarrow \vlastni(X, O), \osoba(X).\\
\operatorname{owned}(O) & \leftarrow \vlastni(X, O), \firma(X).
\end{align*}
\end{problem}


%%%%%%%%%%%%%%%%%%%%%%%%%%%%%%%%%%%%%%%%%%%%%%%%%%%%%%%%%%%%%%%%%%%%%%%%%%%%%%%%%%%%%%%%%%%%%%%%%%%%%%%%%%%%%%%%%%%%%%%%%%%%%%%%%%%%%
\section{Nesprávnych riešení je hocikoľko\dots{} (relačný kalkul, datalog)}

Posolstvom tejto časti je ukázať, že je bezpočet možností, ako napísať chybné riešenie, hoci mnohé chybné riešenia majú k správnemu celkom blízko. Všetky riešenia pochádzajú z istej rozcvičky písanej na treťom cvičení.

Mali by ste trénovať spochybňovanie akéhokoľvek kódu či formálneho zápisu, čo vám príde do rúk (vlastného i cudzieho).
Preštudujte si všetky chybné riešenia a skúste pre každé z nich v prirodzenom jazyku vyjadriť, čo popisuje, alebo aspoň zdôvodniť, prečo je nesprávne.

$EDB = \{\ponuka(Miesto, Akcia), \chce(Clovek, Miesto, Akcia)\}$

\begin{problem}
{\bf Akcie, ktoré sú v ponuke na každom mieste, kde ich chcú aspoň dvaja (a sú aspoň niekde v ponuke).}

Správne riešenie (relačný kalkul, tri ekvivalentné zápisy):
\begin{alignat*}{2}
\{A \mid (\exists M \ponuka(M, A)) & \land \ \ \forall M \forall C_1 \forall C_2        && [C_1 \neq C_2 \land \chce(C_1, M, A)\land \chce(C_2, M, A)\implies \ponuka(M, A)]\}\\
\{A \mid (\exists M \ponuka(M, A)) & \land \neg\exists M \exists C_1 \exists C_2\ \neg  && [C_1 \neq C_2 \land \chce(C_1, M, A)\land \chce(C_2, M, A)\implies \ponuka(M, A)]\}\\
\{A \mid (\exists M \ponuka(M, A)) & \land \neg\exists M \exists C_1 \exists C_2        && [C_1 \neq C_2 \land \chce(C_1, M, A)\land \chce(C_2, M, A)\land \neg \ponuka(M, A)]\}
\end{alignat*}

Správne riešenie (relačný kalkul, tri ekvivalentné zápisy):
\begin{alignat*}{2}
\{A \mid (\exists M \ponuka(M, A)) & \land \ \ \forall M \big[\big(\exists C_1 \exists C_2 && (C_1 \neq C_2 \land \chce(C_1, M, A)\land \chce(C_2, M, A))\big)\implies \ponuka(M, A)\big]\}\\
\{A \mid (\exists M \ponuka(M, A)) & \land \neg\exists M \big[\big(\exists C_1 \exists C_2 && (C_1 \neq C_2 \land \chce(C_1, M, A)\land \chce(C_2, M, A))\big)\land \neg \ponuka(M, A)\big]\}\\
\{A \mid (\exists M \ponuka(M, A)) & \land \neg\exists M \exists C_1 \exists C_2           && [C_1 \neq C_2 \land \chce(C_1, M, A)\land \chce(C_2, M, A)\land \neg \ponuka(M, A)]\}
\end{alignat*}

Správne riešenie (datalog) --- porovnajte so zápismi vyššie:
\begin{align*}
\answer(A) & \leftarrow \ponuka(\_, A), \neg \operatorname{niekdeNeponuka}(A).\\
\operatorname{niekdeNeponuka}(A) & \leftarrow \chce(C_1,M,A), \chce(C_2,M,A) , C_1\neq C_2, \neg \ponuka(M,A).
\end{align*}

Všetky nasledujúce riešenia sú nesprávne (preto tá červená farba). Viete zistiť, v čom sú chybné?

\beginwrong
\begin{align*}
\answer(A) & \leftarrow \ponuka(M, A), \chce(C_1, M, A), \chce(C_2, M, A), C_1 \neq C_2.
\end{align*}

\begin{align*}
\answer(A)& \leftarrow \ponuka(M,A), \operatorname{miestoChcuDvaja}(M).\\
\operatorname{miestoChcuDvaja}(M)& \leftarrow \chce(C_1,M,\_), \chce(C_2,M,\_), C_1\neq C_2.
\end{align*}

\begin{align*}
\operatorname{chcuAsponDvaja}(M, A) & \leftarrow \chce(C_1, M, A), \chce(C_2, M, A), C_1 \neq C_2.\\
\operatorname{zlaAkcia}(M, A) & \leftarrow \operatorname{chcuAsponDvaja}(M, A), \neg \ponuka(M, A).\\
\operatorname{vysledok}(A) & \leftarrow \ponuka(\_, A), \chce(\_, M, \_), \neg \operatorname{zlaAkcia}(M, A).
\end{align*}

\begin{align*}
\operatorname{akciuAsponDvaja}(M, A) & \leftarrow \ponuka(M, A), \chce(C_1, M, A), \chce(C_2, M, A), C_1\neq C_2.\\
\operatorname{nechcuDvaja}(A) & \leftarrow \ponuka(M, A), \neg \operatorname{akciuAsponDvaja}(M, A).\\
\operatorname{odpoved}(A) & \leftarrow \ponuka(\_, A), \neg \operatorname{nechcuDvaja}(A).
\end{align*}

\begin{align*}
\operatorname{pomocna}(A) & \leftarrow \chce(C_1, M, A), \chce(C_2, M, A), C_1\neq C_2, \neg \ponuka(M, A).\\
\operatorname{answer}(A) & \leftarrow \ponuka(\_, A), \operatorname{pomocna}(A).
\end{align*}

\begin{align*}
\operatorname{chcu2}(A)& \leftarrow \chce(C_1,\_,A), \chce(C_2,\_,A), C_1\neq C_2, \neg \operatorname{nievsetky}(A).\\
\operatorname{nievsetky}(A)& \leftarrow \ponuka(\_,A), \neg \ponuka(M,A).
\end{align*}

\begin{align*}
\operatorname{ans}(a) & \leftarrow \ponuka(\_, A), \neg \operatorname{pomoc1}(A).\\
\operatorname{pomoc1}(A) & \leftarrow \ponuka(M,A), \operatorname{pomoc2}(M,A).\\
\operatorname{pomoc2}(M,A) & \leftarrow \chce(P_1, M, A), \chce(P_2, M, A), P_1 \neq P_2, \neg \ponuka(M, A).
\end{align*}

\begin{align*}
\answer(A) & \leftarrow \ponuka(\_,A), \neg \operatorname{chceAviac2neni}(A).\\
\operatorname{chceAviac2neni}(A) & \leftarrow \ponuka(\_,A), \neg \operatorname{chceAmenej2aje}(A).\\
\operatorname{chceAmenej2aje}(A) & \leftarrow \chce(C_1,M,A), \chce(C_2,M,A), C_1 \neq C_2.
\end{align*}

\begin{align*}
\answer(A) & \leftarrow \ponuka(M,A), \neg \operatorname{nechcudvaja}(A, M), \chce(\_, M, A).\\
\operatorname{nechcudvaja}(A, M) & \leftarrow \chce(C_1, M, A), \chce(C_2, M, A), C_1=C_2.
\end{align*}

\begin{align*}
\answer(A)& \leftarrow \ponuka(\_, A), \neg \operatorname{notchcu}(A).\\
\operatorname{notchcu}(A)& \leftarrow \ponuka(\_, A), \chce(C_1, M, \_), \chce(C_2, M, \_), C_1 \neq C_2, \neg \operatorname{not2}(A, M).\\
\operatorname{not2}(A, M)& \leftarrow \chce(\_, M, A).
\end{align*}
\endwrong

\end{problem}

%%%%%%%%%%%%%%%%%%%%%%%%%%%%%%%%%%%%%%%%%%%%%%%%%%%%%%%%%%%%%%%%%%%%%%%%%%%%%%%%%%%%%%%%%%%%%%%%%%%%%%%%%%%%%%%%%%%%%%%%%%%%%%%%%%%%%
\section{Bežné chyby}

V mnohých oblastiach ľudskej činnosti viete dosiahnuť veľmi dobré výsledky čisto tým, že sa vyhýbate chybám (silne odporúčam pohľadať na internete niečo na tému \enquote{invert, always invert}; môžete začať napríklad týmto:
{\small \url{https://www.biznews.com/thought-leaders/1986/06/13/charlie-mungers-speech-to-the-harvard-school-june-1986}}). 

Do istej miery to platí to aj pre tento predmet. Prejdite si preto občas nasledovný zoznam, napríklad zakaždým pri riešení úloh z databázového praktika či pred rozcvičkou.

\begin{enumerate}
\item všeobecné
\begin{itemize}
\item nekorektné argumenty (napr. zámena predikátov/relácií $\navstivil$ a $\vypil$)
\item množina (relačný kalkul, datalog) vs. multimnožina (SQL, relačná algebra)
\item nesprávna interpretácia kvantifikátorov:\\
    $\forall x$ --- pre každý / ľubovoľný / hociktorý prvok $x$, pre všetky $x$\\
    $\exists x$ --- pre nejaký / niektorý / aspoň jeden prvok $x$, existuje $x$
\item zlá interpretácia zadaných relácií (napr. prehliadnutie odlišných cien alkoholu v jednotlivých krčmách)
\end{itemize}


\item relačný kalkul
\begin{itemize}
\item voľné (nekvantifikované) premenné, napr. $Y$ vo výraze $\{X \mid p(X,Y)\}$
\item chýbajúce či nesprávne uzátvorkovanie (logické spojky nemajú štandardne definovanú prioritu ani asociovanie ako operátory v jazyku C)
\item použitie premennej mimo oblasti platnosti (napr. za zátvorkou, vnútri ktorej je kvantifikovaná)
\item porovnávanie predikátov, napr. $p(X) = h(Y)$
\item použitie prvkov mimo jazyka ($\cancel\exists$, $\exists!$)
\item miešanie prvkov datalogu (\_)
\item chýbajúce podmienky na rôznosť premenných, napr. $\exists X\exists Y (p(X) \land p(Y))$ (ekvivalentné $(\exists X) p(X)$)
\end{itemize}

\item datalog
\begin{itemize}
\item použitie pravidiel, ktoré nie sú bezpečné (naučte sa to zakaždým explicitne skontrolovať)
\item nepoužitie pomocného pravidla pri negovaní na miestach, kde je to nutné, napr. $p(X)\leftarrow r(X), \lnot q(X, \_)$
\item nevyužívanie premennej $\_$ (sťažuje čítanie, znemožňuje rýchlu kontrolu, ľahšie sami seba popletiete)
\item nevhodný názov pomocného predikátu (nezrozumiteľný, v rozpore s definíciou predikátu atď.)
\item prebytočný argument v pomocnom predikáte --- mení význam: ak popisujeme vlastnosť krčmy, musí byť táto jediným argumentom, jeden argument navyše spôsobí, že už popisujeme vlastnosť dvojice
\end{itemize}

\item SQL
\begin{itemize}
\item syntaktické chyby
    \begin{itemize}    
    \item \verb|stlpec = tabulka| miesto \verb|stlpec = (SELECT ... FROM tabulka)|
    \item \verb|NOT EXISTS IN| miesto \verb|NOT IN|; \verb|NOT IN tabulka| miesto \verb|NOT IN (SELECT ... FROM tabulka)|
    \item použitie agreg. funkcie na výsledok dotazu, napr. \verb|MAX(SELECT ... FROM ...)|
    \item použitie agreg. funkcie na reláciu, napr. \verb|MAX(tabulka)|
    \end{itemize}
\item chyby vychádzajúce z nepochopenia významu kľúčových slov (nekorektná štruktúra dotazu)
    \begin{itemize}
    \item vymenovanie nepovolených stĺpcov za \verb|SELECT| (všetky musia byť aj za \verb|GROUP BY|)
    \item dvojnásobné agregovanie: \verb|MAX(COUNT(X))|
    \item použitie agregačnej funkcie za \verb|WHERE| (patrí hneď za \verb|HAVING| alebo \verb|SELECT|)
    \item podmienka (najmä za \verb|HAVING|), ktorá nevracia bool, napr. \verb|HAVING MAX(t.X)| miesto korektného výpočtu arg max s dvojnásobným použitím \verb|SELECT|
    \item chybné pokusy o arg max: \verb|SELECT A from t WHERE t.c = MAX(t.c)| --- na nájdenie objektov, pre ktoré sa dosahuje maximum hodnoty, ktorú treba najprv vypočítať grupovaním, treba vždy aspoň dva selecty, napr. \verb|SELECT A from t WHERE t.c = (SELECT MAX(t.c) FROM t)|
    \item používanie \verb|HAVING| (resp. \verb|WHERE|) na definovanie/pomenovanie nového stĺpca --- \verb|HAVING| slúži výlučne na filtrovanie skupín, pomenovania pre stĺpce sa pridávajú za \verb|SELECT| a agregačnú funkciu je nutné zopakovať, napr. \verb|SELECT COUNT(x) AS c FROM t GROUP BY y HAVING COUNT(x) > 2|
    \item nepochopenie faktu, že ak \verb|GROUP BY| chýba, ale za \verb|SELECT| je uvedená agregačná funkcia, bude vo výsledku jediný riadok, pretože všetko je v jednej skupine
    \end{itemize}
\item ďalšie chyby
    \begin{itemize}
    \item chýbajúce podmienky pre join (skúste si to porátať pre každý atribút: ak do vnoreného selectu zvonka vstupuje nejaká hodnota povedzme pre alkohol, pridáva to 1 podmienku; ak sa tam vyskytuje ten atribút v joine povedzme 4x, treba pridať ďalšie 3 podmienky stotožňujúce jednotlivé výskyty --- porovnajte si to s datalogom, tam to funguje analogicky)
    \item viacnásobné použitie relácie bez premenovania, ak sa prekrýva oblasť platnosti (scope) pre jednotlivé použitia
    \item používanie pôvodného mena relácie, ak je premenovaná
    \item neželané duplikáty vo výsledku (alebo za \verb|COUNT|)
    \item neúmyselný nekorelovaný subselect (vnorený dotaz, ktorého výsledok nezávisí od riadka hlavného dotazu, do ktorého je vnorený)
    \item nesprávne či chýbajúce použitie \verb|GROUP BY|
    \item zámena agreg. funkcie, napr. \verb|COUNT| miesto \verb|SUM|
    \end{itemize}
\end{itemize}

\item relačná algebra (\emph{na rozdiel od predošlých aj špecifikuje postup výpočtu})
\begin{itemize}
\item absencia zátvoriek, ktoré by jednoznačne definovali poradie operácií
\end{itemize}
\end{enumerate}



%%%%%%%%%%%%%%%%%%%%%%%%%%%%%%%%%%%%%%%%%%%%%%%%%%%%%%%%%%%%%%%%%%%%%%%%%%%%%%%%%%%%%%%%%%%%%%%%%%%%%%%%%%%%%%%%%%%%%%%%%%%%%%%%%%%%%
\section{Príklady (bez agregácie)}

$EDB = \{\lubi(P, A), \capuje(K, A), \navstivil(I, P, K), \vypil(I, A, M)\}$\\[2mm]
$P$ --- pijan, $A$ --- alkohol, $K$ --- krčma, $M$ --- množstvo ($M > 0$),\\
$I$ --- identifikátor návštevy (v $\navstivil$ sa $I$ vyskytuje len raz

V príkladoch uvádzame správne a niekedy i nesprávne riešenia (červená farba zvýrazňuje chybu alebo označuje nesprávne riešenie ako celok).

\begin{problem}
{\bf Alkoholy, ktoré sa čapujú, ale nikto ich nepil.}\\
\begin{minipage}{0.49\textwidth}
\begin{align*}
\answer(A) &\leftarrow \capuje(\_, A), \lnot\operatorname{niektoPil}(A).\\
\operatorname{niektoPil}(A) &\leftarrow \vypil(\_, A, \_).\\
\end{align*}
\begin{align*}
\pi_A(\capuje)\antijoin\vypil
\end{align*}
\end{minipage}
\begin{minipage}{0.49\textwidth}
\begin{align*}
\beginwrong\answer(A)\endwrong &\leftarrow \capuje(\_, A), \beginwrong\lnot\vypil(\_, A, \_)\endwrong.\\[10mm]
& /* \quad \textit{nájdite 3 chyby} \quad */\\
\answer(A) &\leftarrow \capuje(\_, A), \operatorname{niktoNepil}(A).\\
\beginwrong\operatorname{niktoNepil}(A)\endwrong &\leftarrow \vypil(\_, A_2, \_), A_2\neq \beginwrong A\endwrong.
\end{align*}
\end{minipage}
\end{problem}

\begin{problem}
{\bf Alkoholy, ktoré ľúbi každý pijan, čo niečo ľúbi (a aspoň niekto).}
\begin{align*}
\answer(A) &\leftarrow \lubi(\_, A), \lnot\operatorname{pijanCoNelubi}(A).\\
\operatorname{pijanCoNelubi}(A) &\leftarrow \lubi(P, \_), \lubi(\_, A), \lnot\lubi(P, A).\\
\end{align*}
\begin{align*}
\operatorname{p} & = (\pi_P(\lubi) \times \pi_A(\lubi))\antijoin \lubi\\
\answer & = \pi_A(\lubi) \antijoin \operatorname{p}
\end{align*}
\begin{verbatim}
    SELECT DISTINCT l.A
    FROM lubi l
    WHERE NOT EXISTS (
        SELECT 1 FROM lubi l2
        WHERE NOT EXISTS (
            SELECT 1 FROM lubi l3
            WHERE l3.P = l2.P AND l3.A = l.A
        )
    )
\end{verbatim}
\end{problem}

\begin{problem}
{\bf Alkoholy, ktoré ľúbi každý pijan (ktorý niečo ľúbi), čapujú ich všade (kde niečo čapujú) a niekto ich už pil.}
\begin{align*}
\Big\{A\mid (\exists I\,\exists M\vypil(I, A, M))&\land \lnot \Big[\exists P (\exists A_2\lubi(P, A_2))\land\lnot\lubi(P, A))\Big)\Big]\\
&\land \lnot\Big[\exists K \Big((\exists A_2 \capuje(K, A_2))\land \lnot\capuje(K, A)\Big)\Big]\Big\}
\end{align*}
\begin{align*}
\operatorname{nelubenyNiekym}(A) &\leftarrow \vypil(\_, A, \_), \lubi(P, \_), \lnot\lubi(P, A).\\
\operatorname{necapovanyNiekde}(A) &\leftarrow \vypil(\_, A, \_), \capuje(K, \_), \lnot\capuje(K, A).\\
\answer(A) &\leftarrow \vypil(\_, A, \_), \lnot\operatorname{nelubenyNiekym}(A), \lnot\operatorname{necapovanyNiekde}(A).\\
\end{align*}
\begin{align*}
\operatorname{nelubenyNiekym} & = (\pi_A(\vypil)\times \pi_P(\lubi))\antijoin \lubi\\
\operatorname{necapovanyNiekde} & = (\pi_A(\vypil)\times \pi_K(\capuje))\antijoin \capuje\\
\answer & = ((\pi_A(\vypil) \antijoin \operatorname{nelubenyNiekym})\antijoin \operatorname{necapovanyNiekde}\\
\end{align*}
{\small
\begin{verbatim}
    SELECT DISTINCT v.A
    FROM vypil v
    WHERE NOT EXISTS (SELECT 1 FROM lubi l
                    WHERE NOT EXISTS (SELECT 1 FROM lubi l2
                                        WHERE l2.P = l.P AND l2.A = v.A))
        AND NOT EXISTS (SELECT 1 FROM capuje c
                        WHERE NOT EXISTS (SELECT 1 FROM capuje c2
                                        WHERE c2.K = c.K AND c2.A = v.A))
\end{verbatim}
}
\end{problem}

\begin{problem}
{\bf Krčmy, ktoré navštívil každý pijan, čo niekedy pil a čapujú všetko, čo sa kedy pilo (a aspoň niečo).}
\begin{align*}
\Big\{K\mid (\exists A\capuje(K, A))&\land \lnot \Big[\exists P\Big(\exists I\,\exists A\,\exists M (\vypil(I, A, M)\land \navstivil(I, P, K))\Big)\land \lnot\exists I_2\navstivil(I_2, P, K)\Big]\\
&\land \lnot\Big[\exists A \Big((\exists I\exists M \vypil(I, A, M))\land \lnot\capuje(K, A)\Big)\Big]\Big\}
\end{align*}
\begin{align*}
\answer(K) &\leftarrow \capuje(K, \_), \lnot\operatorname{nenavstivenaNiekym}(K), \lnot\operatorname{necapujeNieco}(K).\\
\operatorname{necapujeNieco}(K) &\leftarrow \capuje(K, \_), \vypil(\_, A, \_), \lnot\capuje(K, A).\\
\operatorname{navstivilNiekedy}(P, K) &\leftarrow \navstivil(\_, P, K).\\
\operatorname{nenavstivenaNiekym}(K) &\leftarrow \capuje(K, \_), \navstivil(I, P, \_), \vypil(I, \_, \_), \lnot\operatorname{navstivilNiekedy}(P, K).\\
\end{align*}
\beginwrong Chybné verzie $\operatorname{nenavstivenaNiekym}$:\endwrong
\begin{align*}
\operatorname{nenavstivenaNiekym}(K) &\leftarrow \navstivil(I, P, \_), \vypil(I, \_, \_), \beginwrong \lnot\navstivil(\_, P, K)\endwrong, \capuje(K, \_).\\
\operatorname{nenavstivenaNiekym}(K) &\leftarrow \navstivil(I, P, \_), \vypil(I, \_, \_), \beginwrong \lnot\navstivil(I_2, P, K), \navstivil(I_2, \_, \_)\endwrong, \capuje(K, \_).
\end{align*}
\end{problem}

\begin{problem}
{\bf Alkoholy, ktoré niekto ľúbi a čapuje ich každá krčma, ktorú nenavštívil Fero.}
\begin{align*}
& \operatorname{bolTam}(P, K) \Longleftrightarrow \exists I \navstivil(I, P, K)\\
\{A\mid \lubi(\_, A) & \land \forall K (\neg\operatorname{bolTam}(\operatorname{fero}, K)\implies \capuje(K, A))\}\\
\{A\mid \lubi(\_, A) & \land \neg\exists K (\neg\operatorname{bolTam}(\operatorname{fero}, K)\land \neg \capuje(K, A))\}
\end{align*}
\begin{align*}
\operatorname{navstivilFero}(K) &\leftarrow \navstivil(I, \operatorname{fero}, K).\\
\operatorname{nevhodna}(K, A) &\leftarrow \capuje(K, \_), \neg\operatorname{navstivilFero}(K), \neg\capuje(K, A), \lubi(\_, A).\\
\answer(A) &\leftarrow \lubi(\_, A), \capuje(K, \_), \neg\operatorname{nevhodna}(K, A).
\end{align*}

\begin{minipage}[t]{0.49\textwidth}
\small
\begin{verbatim}
SELECT DISTINCT l.A
FROM l
WHERE NOT EXISTS (
    SELECT 1
    FROM c
    /* c.K nenavstivil Fero a necapuje l.A */
    WHERE NOT EXISTS (
        SELECT 1
        FROM n
        WHERE n.K = c.K AND n.P = 'Fero'
    ) AND NOT EXISTS (
        SELECT 1
        FROM c c2
        WHERE c2.K = c.K AND c2.A = l.A
    )
);
\end{verbatim}
\end{minipage}
\begin{minipage}[t]{0.49\textwidth}
\small
\begin{verbatim}
CREATE TEMPORARY TABLE krcmyBezFera AS (
    SELECT c.K
    FROM capuje c
    WHERE NOT EXISTS (
        SELECT 1
        FROM navstivil n
        WHERE n.K = c.K AND n.P = 'Fero'
    )
);
SELECT DISTINCT l.A
FROM l
WHERE NOT EXISTS (
    SELECT 1
    FROM krcmyBezFera kbf
    WHERE NOT EXISTS (
        SELECT 1
        FROM c
        WHERE c.K = kbf.K AND c.A = l.A
    )
);
\end{verbatim}
\end{minipage}
\end{problem}

\begin{problem}
{\bf Nájdite pijanov, ktorí navštívili presne tie isté krčmy ako Ignác. Predpokladáme, že Ignác navštívil krčmu aspoň raz.}

\begin{align*}
\answer(P) & \leftarrow \navstivil( \_,P,K), \lnot \operatorname{navIgnac}(P), \lnot \operatorname{nenavIgnac}(P).\\
\operatorname{nav}(P,K) & \leftarrow \navstivil(\_,P,K).\\
\operatorname{navIgnac}(P) & \leftarrow \operatorname{nav}(ignac,K), \lnot \operatorname{nav}(P,K), \navstivil (\_ ,P,\_).\\
\operatorname{nenavIgnac}(P) & \leftarrow \operatorname{nav}(P,K), \lnot \operatorname{nav}(ignac,K).
\end{align*}

\end{problem}


%%%%%%%%%%%%%%%%%%%%%%%%%%%%%%%%%%%%%%%%%%%%%%%%%%%%%%%%%%%%%%%%%%%%%%%%%%%%%%%%%%%%%%%%%%%%%%%%%%%%%%%%%%%%%%%%%%%%%%%%%%%%%%%%%%%%%
\section{Príklady (s agregáciou)}
$EDB = \{\lubi(P, A), \capuje(K, A, C), \navstivil(I, P, K), \vypil(I, A, M)\}$\\[2mm]
$P$ --- pijan, $A$ --- alkohol, $K$ --- krčma, $C$ --- jednotková cena, $M$ --- množstvo ($M > 0$),\\
$I$ --- identifikátor návštevy (v $\navstivil$ sa $I$ vyskytuje len raz)



\begin{problem}
{\bf Krčmy, ktoré čapujú aspoň $5$ alkoholov a nečapujú žiadne dva alkoholy za rovnakú cenu.}\\[5mm]
\begin{minipage}{0.49\textwidth}
\begin{verbatim}
SELECT c.K
FROM capuje AS c
WHERE NOT EXISTS (SELECT 1
                  FROM capuje AS c2
                  WHERE c2.K = c.K AND
                        c2.A <> c.A AND
                        c2.C = c.C)
GROUP BY c.K
HAVING COUNT(c.A) >= 5;
\end{verbatim}
\end{minipage}
\begin{minipage}{0.49\textwidth}
\begin{verbatim}
SELECT c.K
FROM capuje AS c
GROUP BY c.K
HAVING COUNT(c.A) >= 5

EXCEPT

SELECT c1.K
FROM capuje c1, capuje c2
WHERE c1.K = c2.K AND
      c1.A <> c2.A AND
      c1.C = c2.C;
\end{verbatim}
\end{minipage}
\end{problem}



\begin{problem}
{\bf Krčmy, ktoré navštevuje jediný pijan a boli navštívené aspoň $5$-krát.}\\[5mm]
\begin{minipage}{0.49\textwidth}
\begin{verbatim}
SELECT n.K
    FROM n
    GROUP BY n.K
    HAVING COUNT(n.I) >= 5
EXCEPT
SELECT n1.K
    FROM n AS n1, n AS n2
    WHERE n1.K = n2.K AND n1.P <> n2.P;
\end{verbatim}
\end{minipage}
\begin{minipage}{0.49\textwidth}
\begin{verbatim}
SELECT n.K
FROM n
WHERE NOT EXISTS (SELECT 1
                  FROM n AS n2
                  WHERE n2.K = n.K AND
                        n2.P <> n.P)
GROUP BY n.K
HAVING COUNT(n.I) >= 5;
\end{verbatim}
\end{minipage}

\hr

\begin{minipage}{\textwidth}
\begin{verbatim}
SELECT n.K
FROM n
GROUP BY n.K
HAVING COUNT(DISTINCT n.P) = 1 AND COUNT(n.I) >= 5;
\end{verbatim}
\end{minipage}
\end{problem}



\begin{problem}
{\bf Počet krčiem, ktoré čapujú aspoň dva alkoholy z tých, čo ľúbi Fero.}

\begin{verbatim}
feroveOblubeneA = π Alkohol (σ lubi.Pijan = 'fero' (lubi))
c = π Krcma, Alkohol (capuje)
c1 = ρ Alkohol → A1 (c ⨝ feroveOblubeneA)
c2 = ρ Alkohol → A2 (c ⨝ feroveOblubeneA)
γ ; COUNT(DISTINCT Krcma) -> N (π Krcma (σ A1 <> A2 (c1 ⨝ c2)))
\end{verbatim}

\hr

\begin{verbatim}
ferove = (σ P = 'Fero' (lubi)) ⨝ capuje
krcmy = π K (σ p>=2 (γ capuje.K; COUNT(A)→p (ferove))
γ ; COUNT(K) (krcmy)
\end{verbatim}

\hr

\begin{verbatim}
lubiFero = σ P = 'fero' (lubi)
nelubiFero = π A (capuje ▷ lubiFero)
capujePocet = γ K; COUNT(A)->Pocet (capuje ▷ nelubiFero)
γ ; COUNT(K) (σ Pocet>1 (capujePocet))
\end{verbatim}
\end{problem}



\begin{problem}
{\bf Pijani, ktorí ľúbia najviac alkoholov spomedzi tých, čo sa nikde nečapujú (a aspoň jeden taký alkohol ľúbia).}

\begin{minipage}{0.49\textwidth}
\begin{verbatim}
WITH pocty AS (
    SELECT l.P, COUNT(l.A) AS N
    FROM lubi l
    WHERE NOT EXISTS (
        SELECT 1
        FROM capuje c
        WHERE c.A = l.A
    )
    GROUP BY l.P
)

SELECT p.P
FROM pocty p
WHERE p.N = (
    SELECT MAX(p2.N)
    FROM pocty p2
)
\end{verbatim}
\end{minipage}
\begin{minipage}{0.49\textwidth}
\begin{verbatim}
WITH pocetLubenychNecapovanych AS (
    SELECT l.P, COUNT(DISTINCT l.A) AS pocet
    FROM lubi l
    WHERE NOT EXISTS (
        SELECT 1 FROM capuje c
        WHERE l.A = c.A
    )
    GROUP BY l.P
)
SELECT DISTINCT pln.P
FROM pocetLubenychNecapovanych pln
WHERE NOT EXISTS (
    SELECT 1
    FROM pln pln2
    WHERE pln.pocet < pln2.pocet
);
\end{verbatim}
\end{minipage}
\begin{align*}
\operatorname{p} & = \Gamma_{P; \COUNT(\delta A)\rightarrow Pocet}(\lubi\antijoin\capuje)\\
\operatorname{max} & = \rho_{M\rightarrow Pocet}(\Gamma_{\MAX(Pocet)\rightarrow M}(\operatorname{p}))\\
\answer &  = \pi_P (\operatorname{p}\join \operatorname{max})
\end{align*}
\end{problem}



\begin{problem}
{\bf Počet pijanov, ktorí ochutnali každý alkohol čapovaný v aspoň dvoch krčmách (a niečo pili).}

\begin{verbatim}
nv = navstivil ⨝ vypil
c = ρ Krcma → K, Alkohol → A (capuje)
c2 = ρ Krcma → K2, Alkohol → A2 (capuje)
alkohol_v_dvoch = π A (c) ⨝ K != K2 ∧ A = A2 (c2)
nepili_jeden_taky = π Pijan (((π Pijan (nv)) x alkohol_v_dvoch) ▷ nv)
γ ; COUNT(DISTINCT Pijan) (nv ▷ nepili_jeden_taky)
\end{verbatim}

\hr

\begin{verbatim}
nv = navstivil ⨝ vypil
ap = γ Alkohol; count(Krcma)→Pocet (capuje)
nepili_co_mali = π Pijan (((π Pijan (nv)) x π A (σ Pocet >= 2 (ap))) ▷ nv)
γ ; COUNT(DISTINCT Pijan) (nv ▷ nepili_co_mali)
\end{verbatim}

\hr

\begin{verbatim}
SELECT COUNT(DISTINCT n.P)
    FROM navstivil n JOIN vypil v ON n.I = v.I
    WHERE NOT EXISTS (
        SELECT 1
        FROM capuje c1, capuje c2
        WHERE c1.A = c2.A AND c1.K <> c2.K
            AND NOT EXISTS (
                SELECT 1
                FROM navstivil nav 
                JOIN vypil vyp ON nav.I = vyp.I
                WHERE nav.P = n.P AND vyp.A = c1.A
            )
    )
\end{verbatim}
\end{problem}


\begin{problem}
{\bf Pod pijanmi budeme v tejto úlohe rozumieť návštevníkov krčiem.
Úspešný pijan: v každej krčme, čo navštívil, čapujú niečo, čo ľúbi.
Vytvorte zoznam úspešných pijanov a vypočítajte ich podiel medzi všetkými pijanmi.}\\[5mm]
\begin{minipage}[t]{0.49\textwidth}
\begin{verbatim}
WITH neuspesni AS (
    SELECT n.P
    FROM n
    WHERE NOT EXISTS (
        SELECT 1 FROM c
        JOIN l ON l.P = n.P AND l.A = c.A
        WHERE c.K = n.K
    )
)
CREATE TEMPORARY TABLE uspesni AS (
    SELECT DISTINCT P FROM n
    EXCEPT
    SELECT P FROM neuspesni
);
SELECT (SELECT COUNT(P) FROM uspesni)
    / (SELECT COUNT(DISTINCT P) FROM n);
\end{verbatim}
\end{minipage}
\begin{minipage}[t]{0.49\textwidth}
\begin{verbatim}
CREATE TEMPORARY TABLE uspesni AS (
    SELECT DISTINCT P
    FROM n
    WHERE NOT EXISTS (
        SELECT 1
        FROM n n2
        WHERE n2.P = n.P
        AND NOT EXISTS (
            SELECT 1
            FROM c, l
            WHERE c.K = n2.K
            AND l.A = c.A
            AND l.P = n.P
        )
);
SELECT COUNT(DISTINCT u.P) /
       COUNT(DISTINCT n.P)
FROM n, uspesni u;
\end{verbatim}
\end{minipage}
\end{problem}

\begin{problem}
{\bf Dvojice [K, N], kde K je krčma, čo niečo čapuje, a N je počet návštev K, počas ktorých boli vypité všetky alkoholy, čo K čapuje.}\\[5mm]

\begin{verbatim}
SELECT c.K, COUNT(n.I)
FROM capuje c 
    JOIN navstivil n ON n.K = c.K
WHERE NOT EXISTS (
    /* A capovany v n.K, ktory nebol vypity pri n.I */
    SELECT 1
    FROM capuje c2
    WHERE c2.K = n.K AND NOT EXISTS (
        SELECT 1
        FROM vypil v
        WHERE v.I = n.I AND v.A = c2.A        
    )
)
GROUP BY c.K
\end{verbatim}
\end{problem}

\begin{problem}
{\bf Vypočítajte celkové tržby T za predaj najobľúbenejšieho alkoholu A (t.\,j. takého, ktorý ľúbi najviac pijanov --- ak je takých viac, počítame tržby osobitne pre každý z nich); vo výsledku dvojice [A, T].}\\[5mm]
\begin{minipage}{0.49\textwidth}
\begin{verbatim}
WITH alcohol_count AS (
    SELECT A, COUNT(P) as pocet
    FROM l GROUP BY l.A
) CREATE TEMPORARY TABLE najoblub AS (
    SELECT ac.A
    FROM alcohol_count AS ac
    WHERE ac.pocet = (
        SELECT MAX(a.pocet)
        FROM alcohol_count AS a)
);
SELECT v.A, SUM(v.M * c.C)
FROM v JOIN n ON n.I = v.I
    JOIN c ON c.A = v.A AND c.K = n.K
    JOIN najoblub AS na ON na.A = v.A
GROUP BY v.A
\end{verbatim}
\end{minipage}
\begin{minipage}{0.49\textwidth}
\end{minipage}
\end{problem}



\begin{problem}
{\bf Pijani, ktorí už pili a pijú len alkoholy čapované za cenu nižšiu ako priemerná (cez všetky krčmy čapujúce daný alkohol).}\\[5mm]

\begin{verbatim}
priemerne_ceny = γ A; avg(Cena) → PriemernaCena (capuje)
drahe = π K, A (σ Cena ≥ PriemernaCena (capuje ⨝ priemerne_ceny))
(π P (navstivil ⨝ vypil)) ▷ ((navstivil ⨝ vypil) ⨝ drahe)
\end{verbatim}

\begin{verbatim}
WITH priemery AS (
    SELECT c.A, AVG(c.C) AS priemer
    FROM capuje c
    GROUP BY c.A
)
SELECT n1.P
FROM navstivil n1, vypil v1
WHERE n1.I = v1.I AND NOT EXISTS (
    SELECT 1
    FROM navstivil n2, vypil v2, priemery p, capuje c
    WHERE n2.I = v2.I AND c.K = n2.K AND c.A = v2.A AND n2.P = n1.P AND p.A = c.A AND c.C >= p.priemer
)
\end{verbatim}

\beginwrong
\begin{verbatim}
WITH priemery AS ...
SELECT DISTINCT n1.P
FROM navstivil n1, vypil v1, priemery
WHERE n1.I = v1.I AND NOT EXISTS (
    SELECT 1
    FROM navstivil n2 JOIN vypil v2, capuje c2
    WHERE n2.I = n1.I AND c2.K = n2.K AND c2.A = v2.A AND n2.P = n1.P
        AND priemery.A = v2.A AND c2.C >= priemery.priemer
);
\end{verbatim}
\endwrong

\end{problem}



\begin{problem}
{\bf Krčmy, ktoré niečo čapujú, ale väčšina návštevníkov v nich nič neľúbi (ak pijan navštívi krčmu viackrát, rátame ho medzi návštevníkov len raz).}\\[5mm]

\begin{verbatim}
WITH lubi_v_krcme AS (
    SELECT c.K, COUNT(DISTINCT l.P) AS pocet
    FROM capuje c
    JOIN navstivil n ON c.K = n.K
    JOIN lubi l ON n.P = p.P AND c.A = l.A
    GROUP BY c.K
),
WITH nelubi_v_krcme AS (
    SELECT c.K, COUNT(DISTINCT n.P) AS pocet
    FROM capuje c
    JOIN navstivil n ON c.K = n.K
    WHERE NOT EXISTS(
        SELECT 1
        FROM capuje c2
        JOIN lubi l ON c2.A = l.A AND n.P = l.P
        WHERE c.K = c2.K
    )
    GROUP BY c.K
)
SELECT l.K
FROM lubi_v_krcme l
JOIN nelubi_v_krcme n ON l.K = n.K AND l.pocet < n.pocet;
\end{verbatim}
\end{problem}



\begin{problem}
{\bf Nájdite všetkých pijanov, ktorí pili aspoň raz a nie je pravda, že by v nejakej krčme celkovo prepili viac ako v niektorých dvoch iných dokopy.}\\[5mm]

\begin{minipage}[t]{0.49\textwidth}
\begin{verbatim}
CREATE TEMPORARY TABLE prepil (
    SELECT n.P, n.K, SUM(v.M * c.C) AS Suma
    FROM navstivil n, capuje c, vypil v
    WHERE n.I = v.I AND c.A = v.A AND n.K = c.K
    GROUP BY n.P, n.K
)

CREATE TEMPORARY TABLE dvojice (
    SELECT p1.P, p1.K AS K1, p2.K AS K2,
        (p1.Suma + p2.Suma) AS Sucet
    FROM prepil p1, prepil p2
    WHERE p1.P = p2.P AND p1.K != p2.K
)

CREATE TEMPORARY TABLE zli_pijani (
    SELECT DISTINCT p.P
    FROM prepil p, dvojice d
    WHERE p.P = d.P 
        AND p.K != d.K1 AND p.K != d.K2
        AND p.Suma > p.Sucet
)

SELECT DISTINCT n.P
FROM navstivil n, vypil v
WHERE n.I = v.I
EXCEPT (SELECT P from zli_pijani)
\end{verbatim}
\end{minipage}
\begin{minipage}[t]{0.49\textwidth}
\begin{verbatim}
CREATE TEMPORARY TABLE P_celkove_prepil_v_K AS (
    SELECT n.P, n.K, SUM(v.M*c.C) AS cena 
    FROM navstivil AS n, capuje AS c, vypil AS v
    WHERE n.I=v.I AND 
          n.K=c.K AND 
          c.A=v.A
    GROUP BY n.P, n.K
);

SELECT n.P
FROM n, v
WHERE n.I = v.I AND NOT EXISTS (
    SELECT 1
    FROM P_celkove_prepil_v_K AS b1,
        P_celkove_prepil_v_K AS b2,
        P_celkove_prepil_v_K AS b3
    WHERE b1.cena > b2.cena + b3.cena AND
        b2.K <> b3.K AND
        b1.P = n.P AND
        b2.P = n.P AND
        b3.P = n.P
)
\end{verbatim}
\end{minipage}

\begin{align*}
\answer(P) &\leftarrow \navstivil(I,P,\_), \vypil(I,\_,\_), \lnot \operatorname{mocnyUcet}(P).\\
\operatorname{mocnyUcet}(P) &\leftarrow \operatorname{ucet}(P,K_1,U_1), \operatorname{ucet}(P,K_2,U_2), \operatorname{ucet}(P,K_3,U_3),\\
 & \qquad K_1\neq K_2, K_2\neq K_3, K_1\neq K_3, U_1 > U_2 + U_3.\\
\operatorname{ucet}(P,K,U) &\leftarrow subtotal(\operatorname{p}(\_,P,K,\_,C,M), [P,K], [U=\SUM(C*M)]).\\
\operatorname{p}(I,P,K,A,C,M) &\leftarrow \navstivil(I,P,K), \vypil(I,A,M), \capuje(K,A,C).\\
\end{align*}
\end{problem}



\begin{problem}
{\bf Krčmy, čo niečo čapujú a v ktorých je viac ako polovica tržieb dosiahnutá za alkoholy, ktoré ľúbi aspoň niekto a ľúbia ich všetci pijani (čo niečo ľúbia) okrem práve jedného.}


Najprv alkoholy, ktoré ľúbi aspoň niekto a ľúbia ich všetci pijani (čo niečo ľúbia) okrem práve jedného.

\begin{align*}
\operatorname{alkoholyCoLubiaVsetciOkrem1}(K) &\leftarrow \lubi(\_, A), \lubi(P, \_), \lnot \lubi(P, A), \lnot \operatorname{niektoInyNelubi}(A, P).\\
\operatorname{niektoInyNelubi}(A, P) &\leftarrow \lubi(\_, A), \lubi(P, \_), \lubi(P2, \_), P2\neq P, \lnot \lubi(P2, A).
\end{align*}

\begin{align*}
\operatorname{alkoholyCoLubiaVsetciOkrem1}(K) &\leftarrow \lubi(\_, A), \lubi(P, \_), \lnot \lubi(P, A), \lnot \operatorname{dvajaNelubia}(A).\\
\operatorname{dvajaNelubia}(A) &\leftarrow \lubi(\_, A), \lubi(P1, \_), \lubi(P2, \_), P1\neq P2, \lnot\lubi(P1, A), \lnot \lubi(P2, A).\\
\end{align*}

\begin{minipage}[t]{0.49\textwidth}
\begin{verbatim}
CREATE TEMPORARY TABLE spravne_alkoholy as (
    SELECT l.A
    FROM lubi l
    WHERE EXISTS (
        SELECT p.P
        FROM lubi p
        WHERE NOT EXISTS (
            SELECT 1
            FROM lubi
            WHERE lubi.P = p.P AND lubi.A = l.A
        )
    ) AND NOT EXISTS (
        SELECT p1.P, p2.P
        FROM lubi p1, lubi p2
        WHERE p1.P <> p2.P AND NOT EXISTS (
            SELECT 1
            FROM lubi
            WHERE lubi.P = p1.P AND lubi.A = l.A
        ) AND NOT EXISTS (
            SELECT 1
            FROM lubi
            WHERE lubi.P = p2.P AND lubi.A = l.A
        )
    )
);
\end{verbatim}
\end{minipage}
\begin{minipage}[t]{0.49\textwidth}
\begin{verbatim}
CREATE TEMPORARY TABLE spravne_alkoholy AS (
    SELECT a.A
    FROM lubi a, lubi p
    WHERE NOT EXISTS (
        SELECT 1
        FROM lubi l
        WHERE l.A = a.A AND l.P = p.P 
    )
    GROUP BY a.A
    HAVING COUNT(DISTINCT p.P) = 1
);
\end{verbatim}
\end{minipage}

A teraz už len zrátame tržby.

\begin{minipage}[t]{0.6\textwidth}
\begin{verbatim}
/* trzby za vsetky alkoholy */
WITH vsetky_trzby as (
    SELECT c.K, SUM(n.Mnozstvo * c.Cena) AS trzba
    FROM capuje c, navstivil n, vypil v
    WHERE n.I = v.I AND c.K = n.K AND c.A = v.A
    GROUP BY c.K
),
/* trzby za spravne alkoholy */
spravne_trzby as (
    SELECT c.K, SUM(n.Mnozstvo * c.Cena) AS trzba
    FROM capuje c, navstivil n, vypil v
    WHERE n.I = v.I AND c.K = n.K AND c.A = v.A
        AND EXISTS (
            SELECT 1
            FROM spravne_alkoholy sa
            WHERE v.A = sa.A
        )
    GROUP BY c.K
)
\end{verbatim}
\end{minipage}
\begin{minipage}[t]{0.4\textwidth}
\begin{verbatim}
SELECT c.K
FROM capuje c
WHERE EXISTS (
    SELECT 1
    FROM spravne_trzby s, vsetky_trzby v
    WHERE s.K = c.K AND v.K = c.K 
        AND 2 * s.trzba > v.trzba
)
\end{verbatim}
\end{minipage}

\medskip

Podstatne stručnejšie v relačnej algebre a datalogu (premyslite si, kde by sa ešte dal skrátiť):
\begin{verbatim}
nelubi_prave_jeden = π Alkohol (σ N=1 (γ A; count(P)→N (((π P lubi) x (π A lubi)) ▷ lubi)))
nvc = (navstivil ⨝ vypil) ⨝ capuje
trzby_npj = π K, Tnpj ← M*Cena ( (γ K, A, Cena; SUM(Mnozstvo)→M (nvc ⨝ nelubi_prave_jeden) )
trzby_celkovo = π K, T ← M*Cena ( (γ K, A, Cena; SUM(Mnozstvo)→M (nvc) )
π K (σ Tnpj > T/2 (trzby_celkovo ⨝ trzby_npj))
\end{verbatim}

\begin{verbatim}
krcma_group(I,K,A,C,M) :- navstivil(I,P,K), vypil(I,A,M), capuje(K,A,C).
trzby_celkovo(K,T) :- subtotal(krcma_group(_,K,_,C,M), [K], [T=SUM(C*M)]).
alkoholy_lubia_vsetci(A) :- lubi(_,A), \+zly_pijan(A, P2), lubi(P2,_),\+lubi(P2,A).
zly_pijan(A,P2) :- lubi(P,_),lubi(_,A), \+lubi(P,A), lubi(P2,_), \+P=P2.
krcma_group2(I,K,A,C,M) :- navstivil(I,P,K), vypil(I,A,M), capuje(K,A,C), alkoholy_lubia_vsetci(A).
trzby_za_alkoholy(K,T) :- subtotal(krcma_group2(_,K,_,C,M), [K], [T=SUM(C*M)]).
result(K) :- capuje(K,_,_), trzby_celkovo(K,T), trzby_za_alkoholy(K,T2), T2>(T/2).
\end{verbatim}
\end{problem}



\begin{problem}
{\bf Dvojice [K, A] také, že krčma K čapuje alkohol A za cenu nižšiu ako je priemerná cena A (cez všetky krčmy, čo A čapujú), a každý pijan, ktorý niekde pil A za cenu vyššiu ako v K, ho pil aj v K.}\\[5mm]

\begin{verbatim}
priemernaCenaAlkoholu(A, PC) :- subtotal(capuje(K, A, C), [A], [PC = AVG(C)]).
asponRazVypilAvK(P, A, K) :- n(Id, P, K), vypil(Id, A, _).
pijanNiekdeVypilDrahsiAleNikdyNevypilAvK(K, A, C) :- capuje(K, A, C), capuje(K2, A, C2), C2 > C,
        n(I, P, K2), vypil(I, A, _), \+asponRazVypilAvK(P, A, K).
answer(K, A) :- capuje(K, A, C), priemernaCenaAlkoholu(A, PriemernaCenaA), C < PriemernaCenaA,
        \+pijanNiekdeVypilDrahsiAleNikdyNevypilAvK(K, A, C).
\end{verbatim}

\hr

\begin{verbatim}
WITH priemerne_ceny AS (
    SELECT A, AVG(Cena) as p
        FROM capuje
        GROUP BY A
)
SELECT c.K, c.A
FROM capuje c, priemerne_ceny p
WHERE p.A = c.A AND c.Cena < p.p AND NOT EXISTS (
    /* neexistuje taky P, ktory vypil c.A niekde drahsie a zaroven ho nevypil v c.K */
    SELECT n.P
    FROM navstivil n, vypil v, capuje c2
    WHERE n.I = v.I AND v.A = c.A AND c2.K = n.K AND c2.A = c.A AND c2.Cena > c.Cena
        AND NOT EXISTS (
            SELECT 1 FROM vypil v2, navstivil n2
            WHERE v2.I = n2.I AND n2.P = n.P AND n2.K = c.K AND v2.A = c.A
        )
)
\end{verbatim}
\end{problem}


\begin{problem}
Rozšírme záznamy návštev krčiem o začiatok a koniec návštevy, čiže $\navstivil(I, P, K, Od, Do)$. Predpokladáme, že návštevy jedného pijana sa neprekrývajú ani v koncových bodoch. Ak v jednom momente pijan prichádza a iný odchádza, sú v tom momente obaja v krčme.

{\bf Pre každú navštívenú krčmu nájdite maximálny počet pijanov, ktorí v nej boli prítomní naraz.}

\begin{verbatim}
-- Vsetky casy, v ktorych sa nieco stalo
vc = (ρ Cas<-Od (π Od navstivil)) ∪ (ρ Cas<-Do (π Do navstivil))

-- vsetky navstevy, ktore obsahuju dany cas pre kazdy cas
casnavsteva = σ Cas >= Od and Cas <= Do (vc ⨯ navstivil)

caskrcmapocet = γ Cas, K; COUNT(I)->Pocet (casnavsteva)
γ K; max(Pocet)->Max (caskrcmapocet)
\end{verbatim}

\begin{verbatim}
γ K; max(N)→M (γ T, K; count(I)→N ( (navstivil ⨝ T≥Od ∧ T≤Do (ρ Od→T (π Od (navstivil))))))
\end{verbatim}
\end{problem}



\begin{problem}
Rozšírme záznamy návštev krčiem o začiatok a koniec návštevy, čiže $\navstivil(I, P, K, Od, Do)$. Predpokladáme, že návštevy jedného pijana sa neprekrývajú ani v koncových bodoch.

{\bf V jednej krčme je kvôli koronavírusu možné mať naraz najviac $20$ zákazníkov. Pre každú krčmu zistite súčet dĺžok časových úsekov, počas ktorých bolo toto pravidlo porušené.}

\begin{verbatim}
WITH okamihy AS (
    SELECT DISTINCT Cas
    FROM (
        SELECT n.Od AS Cas
        FROM navstivil n
        UNION
        SELECT n.Do AS Cas
        FROM navstivil n
    )
),
najmensie_intervaly AS (
    -- cely cas rozdeleny na najkratsie zmysluplne disjunktne casove useky
    SELECT o1.Cas AS Od, o2.Cas AS Do
    FROM okamihy o1, okamihy o2
    WHERE o1.Cas < o2.Cas
    AND NOT EXISTS (
        SELECT 1
        FROM okamihy o3
        WHERE o1.Cas < o3.Cas AND o3.Cas < o2.Cas
    )
),
pocty_v_intervaloch AS (
    -- kolko ludi bolo medzi Od-Do v Krcme
    SELECT n.Krcma, i.Od, i.Do, COUNT(n.Id) as Pocet
    FROM navstivil n, najmensie_intervaly i
    WHERE n.Od <= i.Od
    AND n.Do >= i.Do
    GROUP BY n.Krcma, i.Od, i.Do
)
SELECT p.Krcma, SUM(p.Do - p.Od) AS Trvanie
FROM pocty_v_intervaloch p
WHERE p.Pocet > 20
GROUP BY p.Krcma
\end{verbatim}

\begin{verbatim}
CREATE TEMPORARY TABLE krcma_intervaly AS (
	(
		SELECT	n.K, n.Od AS Int
		FROM	navstivil AS n
	)
	UNION
	(
		SELECT	n.K, n.Do AS Int
		FROM	navstivil AS n
	)
);

CREATE TEMPORARY TABLE krcma_ocislovane_interavaly AS(
	SELECT	ki.K, ki.Int, RANK() OVER (ORDER BY ki.Int) as rank
	FROM	krcma_intervaly AS ki
);

CREATE TEMPORARY TABLE krcma_zaciatok_koniec_pocet AS (
	SELECT	koi1.K, koi1.Int as zac, koi2.Int as kon, COUNT(n.I) as pocet
	FROM	krcma_ocislovane_intervaly koi1, krcma_ocislovane_intervali koi2 , navstivil AS n
	WHERE	koi1.K = koi2.K AND koi2.K = n.K AND 
		koi1.rank = koi2.rank - 1 AND 
		n.Od <= koi1.Int AND n.Do >= koi2.Int
	GROUP BY koi1.K, koi1.Int, koi2.Int
);

SELECT	kzkp.K, SUM(kzkp.kon - kzkp.kon)
FROM	krcma_zaciatok_koniec_pocet kzkp
WHERE	kzkp.pocet >=20
GROUP BY kzkp.K
\end{verbatim}

\begin{verbatim}
navstivil_v_case(K,T,P) :- navstivil(_,P,K,Od,Do), Od < T, Do > T.
navstivilo_v_cas(K,T,C) :- subtotal(navstivil_v_case(K,T,P),[K,T],[C=count(P)]).
nad_20_v_case(K,T) :- navstivilo_v_cas(K,T,C), C > 20.
nie_je_najblizsii_odchod(K,Od,Do) :- navstivil(_,_,K,_,T), \+nad_20_v_case(K,T), Od < T, Do > T.
nad_20_cas(K,C) :- navstivil(_,_,K,Od,_), nad_20_v_case(K,Od), navstivil(_,_,K,_,Do), Od < Do, \+nad_20_v_case(K,Do), \+nie_je_najblizsii_odchod(K,Od,Do), C is Do-Od.
answer(K,C) :- subtotal(nad_20_cas(K,T),[K],[C=sum(T)]).
\end{verbatim}

\end{problem}



%%%%%%%%%%%%%%%%%%%%%%%%%%%%%%%%%%%%%%%%%%%%%%%%%%%%%%%%%%%%%%%%%%%%%%%%%%%%%%%%%%%%%%%%%%%%%%%%%%%%%%%%%%%%%%%%%%%%%%%%%%%%%%%%%%%%%
\section{Zadania (bez agregácie)}

\begin{enumerate}
\item 
$EDB = \{osoba(A), pozna(Kto, Koho)\}$
\begin{itemize}
    \item osoby, ktoré poznajú sysľa
    \item osoby, ktoré nepoznajú nikoho (žiadne iné osoby)
    \item osoby, ktoré majú aspoň dvoch známych (osoby)
    \item osoby, ktoré pozná presne jedna osoba
    \item osoby, ktoré poznajú iba Jožka
    \item osoby, ktoré poznajú všetkých známych svojich známych
    \item osoby, ktoré majú všetky vzťahy symetrické
\end{itemize}

\item $EDB = \{integer(X), multiply(X, Y, Result)\}$
\begin{itemize}
    \item hodnota $5\cdot 4$
    \item množina párnych čísel
    \item množina nepárnych čísel
    \item dvojice nesúdeliteľných čísel
\end{itemize}

\item $EDB = \{clovek(Meno), rieka(R), vstupil(Id\_vstupu, Meno, Rieka)\}$
\begin{itemize}
    \item ľudia, ktorí nevstúpili dvakrát do tej istej rieky
    \item ľudia, ktorí do aspoň jednej rieky vstúpili presne raz
    \item rieky, do ktorých vstupovali najviac dvaja rôzni ľudia
    \item dvojice [C, R], kde C je človek, ktorý vstúpil do všetkých riek okrem R a do R nie
\end{itemize}

\item $EDB=\{hodnotenie(Student, Predmet, Znamka)\}$ (dvojice [Student, Predmet] sú unikátne, čiže študent má z daného predmetu najviac jednu známku)
\begin{itemize}
    \item študenti, ktorí majú aspoň 3 známky A
    \item študenti, ktorí boli známkovaní aspoň raz, ale nemajú Fx
    \item dvojice študentov, ktorí majú rovnaké známky zo všetkých predmetov, na ktorých boli obaja hodnotení
    \item študenti, ktorí majú z aspoň dvoch predmetov rovnaké známky ako nejaký iný študent
\end{itemize}

\item
$EDB = \{blysti(Vec), zlate(Vec)\}$
\begin{itemize}
    \item nie je všetko zlato, čo sa blyští    
    \item veci, ktoré sú zlaté, ale neblyštia sa
    \item veci, ktoré ani nie sú zlaté, ani sa neblyštia
\end{itemize}

\item
$EDB = \{kope(Kto, Komu, Jama), padol(Meno, Jama)\}$
\begin{itemize}
    \item všetci, čo druhému jamu kopú, ale sami do nej padli
    \item tí, čo žiadnu jamu nekopú, ale do nejakej padli
    \item tí, čo padli do každej jamy, ktorú kopali
    \item jamy, kto ktorých padol každý, kto nejakú jamu kopal
\end{itemize}

\item
$EDB = \{ponuka(Miesto, Akcia), chce(Clovek, Miesto, Akcia)\}$
\begin{itemize}
    \item exkluzívne akcie, ktoré sa ponúkajú na jedinom mieste a niekto ich tam chce
    \item akcie, ktoré sa neponúkajú na žiadnom mieste, kde ich niekto chce, ale niekde inde áno
    \item akcie, ktoré sa ponúkajú všade, kde ich niekto chce
    \item miesta, kde sa ponúkajú všetky akcie, ktoré chce vtákopysk
\end{itemize}

\item $EDB = \{part(Item), component(Item, Subitem)\}$
\begin{itemize}
    \item súčiastky, ktoré sú zložené z aspoň dvoch komponentov
    \item atomické súčiastky (z ničoho sa neskladajú)
    \item všetky komponenty motora (či už atomické alebo nie)
    \item atomické súčiastky potrebné na zloženie televízora
\end{itemize}

\item $EDB = \{citatel(CitatelID, Meno, DatumPrihlasenia), kniha(KnihaID, Nazov, Autor),$\\
\hspace*{1cm}  $vypozicka(CitatelID, KnihaID, DatumPozicania, DatumVratenia)$\\
(dátumy sú vyjadrené ako počet sekúnd od 1. 1. 1970; DatumVratenia je null, ak kniha nebola vrátená)
\begin{itemize}
    \item trojice [M, K, DatumPozicania], kde K je názov knihy, ktorú čitateľ s menom M ešte nevrátil
    \item názvy kníh, ktoré si ešte nikto nepožičal
    \item mená čitateľov, ktorí si už požičali všetky knihy
    \item mena citateľov, ktorí vrátili všetko, čo si požičali
    \item dvojice [M, K], kde K je názov knihy, ktorú si čitateľ M požičal práve dvakrát
    \item dvojice [M, K], kde K je názov knihy, ktoru si čitateľ M požičal skôr ako Pipi Dlhá Pančucha
    \item čitatelia, ktorí si požičali všetky Haškove knihy z knižnice skôr, než niektorú z nich vrátili
    \item mená čitateľov, ktorí si požičali aspoň 2 rôzne knihy, pričom všetky ich požičané knihy boli od toho istého autora
    \item dvojice [M, K], kde K je prvá požičaná kniha daného čitateľa
    \item mená autorov, ktorých knihu si požičal každý čitateľ (bez duplikátov)
    \item dvojice [M, N], kde N je počet dní čitateľovej najdlhšej výpožičky (pozor na null)
    \item čitatelia, ktorí si nepožičali nič v deň, keď sa zaregistrovali, a naraz majú požičanú vždy nanajvýš jednu knihu
    \item čitatelia, ktori si čítajú knihy \enquote{po autoroch}: ak raz precitaju knihu od nejakého autora, čítajú len knihy tohto autora, až kým neprečítajú všetky jeho knihy, ktoré sú v knižnici
\end{itemize}



\item $EDB=\{lubi(Pijan, Alkohol), capuje(Krcma, Alkohol),$\\
\hspace*{1cm} $navstivil(Id, Pijan, Krcma), vypil(Id, Alkohol, Mnozstvo)\}$
\begin{itemize}
    \item krčmy, kde sa čapuje pivo a nič iné
    \item krčmy, kde sa pije pivo a nič iné
    \item pijani, ktorí ľúbia len rum
    \item pijani, ktorí ľúbia práve jeden alkohol
    \item pijani-štamgasti, ktorí doteraz navštívili jedinú krčmu
    \item pijani, ktorí niekedy niekde vypili viac ako pol litra jedného alkoholu
    \item pijani, ktorí nikdy neodolali rumu (pili ho pri každej návšteve krčmy, v ktorej ho čapujú)
    \item pijani, ktorí ľúbia aspoň niečo a ľúbia každý alkohol, ktorý je čapovaný v aspoň dvoch krčmách
    \item pijani, ktorí v aspoň dvoch krčmách ľúbia všetky tam čapované alkoholy
    \item pijani, ktorí ľúbia aspoň jeden taký alkohol, ktorý čapuje každá krčma, v ktorej ten pijan niečo vypil
    \item pijani, ktorí pili rum a pili ho v každej krčme, kde ho čapujú, s výnimkou najviac jednej
    \item pijani, ktorí pri niektorej svojej návšteve krčmy vytvorili doteraz platný rekord v pití vodky v danej krčme
    \item alkoholy, ktoré ľúbi každý, čo niečo ľúbi (a aspoň niekto), ale nepili sa v každej krčme, čo niečo čapuje a kde sa niečo pilo
    \item pijanov, ktorí vypili len tie alkoholy, ktoré vypil pijan Felix (t.j. hľadaní pijani vypili nejakú neprázdnu podmnožinu alkoholov,
            ktoré vypil pijan Felix; a okrem tých alkoholov nevypili žiadne iné)
    \item alkoholy, pre ktoré platí, že ak ten alkohol niektorý pijan niekedy vypil, tak ho ten pijan vypil pri každej svojej návšteve krčmy
            (vo výsledku majú byť aj alkoholy, ktoré nikto nikdy nevypil)
    \item krčmy, pre ktoré platí: hľadanú krčmu nenavštívil žiaden pijan, ktorý ľúbi všetky alkoholy, ktoré tá krčma čapuje (predpokladajte, že každá krčma čapuje nejaký alkohol) 
    \item všetkých takých pijanov, ktorí neľúbia pivo ani borovičku; a zároveň sa dôsledne vyhýbajú návštevám takým krčiem, v ktorých sa čapuje len pivo alebo borovička;
            a zároveň nikdy pivo ani borovičku nevypili
    \item alkoholy, ktoré ľúbia len tí pijani, ktorí nikdy nenavštívili krčmu Wasa
    \item pijani, ktorí z každého alkoholu A, čo ľúbia, v aspoň jednej krčme aspoň raz vypili na jedno posedenie viac, ako hocikto iný, čo tam niečo pil
    \item dvojice [P, K] také, že pijan P pri každej návšteve krčmy K vypil niektorý z alkoholov, ktoré ľúbi
            (pri rôznych návštevách mohol vypiť rôzne obľúbené alkoholy, chceme len dvojice, kde P niekedy navštívil K) 
    \item dvojice [P, A], ktoré hovoria, ktoré alkoholy A pijan P vypil pri každej svojej návšteve krčmy (abstinenti nemajú byť vo výsledku)
    \item dvojice [P, A] také, že pijan P ľúbi alkohol A, a zároveň každá krčma, v ktorej P vypil A, čapuje alkohol A lacnejšie než ktorákoľvek iná krčma, ktorá čapuje A
    \item dvojice [P, A] také, že pijan P ľúbi alkohol A a ešte také dva ďalšie (navzájom rôzne) alkoholy, že pri každej návšteve krčmy,
            pri ktorej P vypil A, vypil aj niektorý z týchto dvoch ďalších alkoholov
    \item dvojice [A, K] také, že alkohol A čapovaný v krčme K vypil (pri aspoň jednej návšteve) každý pijan, ktorý K niekedy navštívil
    \item dvojice [P, A] také, že pijan P ľúbi alkohol A; a v každej krčme, ktorá čapuje alkohol A, vypil P počas niektorej
            návštevy viacej alkoholu A než ktorýkoľvek iný pijan počas jednej návštevy (teda P je rekordérom v pití A na jedno posedenie v každej krčme, ktorá A čapuje)
    \item pijanov, ktorí každý akt vypitia alkoholu urobili v jednej z krčiem, kde je ten alkohol najlacnejší (abstinenti nemajú byť vo výsledku)    
    \item dvojice [K, A] také, že krčma K čapuje alkohol A, a zároveň každý pijan, ktorý ľúbi alkohol A, ho vypil pri niektorej návšteve krčmy K (čiže tú krčmu aj navštívil)
\end{itemize}

\end{enumerate}


%%%%%%%%%%%%%%%%%%%%%%%%%%%%%%%%%%%%%%%%%%%%%%%%%%%%%%%%%%%%%%%%%%%%%%%%%%%%%%%%%%%%%%%%%%%%%%%%%%%%%%%%%%%%%%%%%%%%%%%%%%%%%%%%%%%%%
\section{Zadania (s agregáciou)}

\begin{enumerate}

\item $EDB=\{lubi(Pijan, Alkohol), capuje(Krcma, Alkohol, Cena),$\\
\hspace*{1cm} $navstivil(Id, Pijan, Krcma), vypil(Id, Alkohol, Mnozstvo)\}$
\begin{itemize}
    \item pijani, ktorí ľúbia aspoň 10 rôznych alkoholov
    \item alkoholy, ktoré boli vypité v krčme Stein v celkovom množstve väčšom ako 20
    \item dvojice [A, Suma], ktoré popisujú množstvo alkoholu A vypitého v krčme Carlton (vo výsledku len tie, čo sa niekedy pili)
    \item dvojice [P, Pocet], ktoré hovoria, v koľkých krčmách prepil pijan P aspoň 10 EUR počas niektorej (jednej) návštevy
    \item trojice [P, A, Pocet], ktoré hovoria, pri koľkých návštevách pijan P vypil alkohol A (netreba nájsť trojice s počtom 0)
    \item pijani, ktorí sú v niektorej krčme lokálnymi šampiónmi v pití rumu (t.j. hľadaný pijan v aspoň jednej krčme vypil dokopy viacej rumu než ľubovoľný iný pijan)
    \item dvojice [P, Suma], ktoré hovoria, koľko peňazí pijan P celkovo prepil v krčmách, ktoré čapujú len alkoholy, ktoré P neľúbi (dvojice s nulovou sumou nemajú byť vo výsledku)
    \item trojice [P, nK, nA], kde nK je počet rôznych krčiem, ktoré pijan P navštívil, a nA je počet rôznych alkoholov, ktoré pijan P vypil (nechceme trojice, kde $nK=nA=0$) 
    \item trojice [K, A, m], kde m je celkové množstvo alkoholu A vypitého v krčme K (chceme vo výsledku každú dvojicu K, A, kde K čapuje A)
    \item dvojice [K, A] také, že alkohol A sa v krčme K vypil v celkovom množstve väčšom ako 50
    \item trojice [P, K, Pocet], ktoré hovoria, pri koľkých návštevách pijan P vypil v krčme K aspoň 5 borovičiek na jedno posedenie (trojice s nulovým počtom nás nezaujímajú)
    \item dvojice [A, M] také, že M je mediánom ceny alkoholu A cez všetky krčmy, ktoré alkohol A čapujú
    \item dvojice [K, Suma], ktoré hovoria, koľko peňazí v krčme K celkovo prepili pijani, ktorí tú krčmu navštívili viac než stokrát (dvojice s nulovou sumou nemajú byť vo výsledku)
    \item trojice [P, K, Priemer] ktoré hovoria, koľko peňazí utratil pijan P v priemere pri jednej návšteve krčmy K (trojice s nulovým priemerom nemajú byť vo výsledku)
    \item všetky dvojice [K, $R$], kde K je krčma, ktorú niekto navštívil,
        a $R\in [0,1]$ je podiel sklamaných pijanov, čiže podiel počtu pijanov, ktorí K navštívili, ale neľúbia žiaden alkohol, ktorý K čapuje,
        k celkovému počtu pijanov, ktorí K navštívili
\end{itemize}

\end{enumerate}


\end{document}

\begin{problem}
{\bf Nájdite krčmy, ktoré už boli navštívené, a chodia do nich výlučne pijani, ktorí pili aspoň
dvakrát a zakaždým prepijú rovnakú sumu (pri akejkoľvek svojej návšteve ľubovoľnej krčmy).}\\[5mm]

\begin{minipage}{0.49\textwidth}
\begin{verbatim}
WITH navstevy AS (
    SELECT P, SUM(v.M * c.C) AS S
    FROM n JOIN v ON n.I = v.I
        JOIN c ON c.K = n.K AND c.A = v.A
    GROUP BY n.I
)
SELECT DISTINCT krcma 
FROM navstivil AS n1
WHERE pijan IN (
    SELECT n.pijan 
    FROM navstivil AS n 
        JOIN vypil AS v ON n.id = v.id
    WHERE NOT EXISTS (
        SELECT 1
        FROM navstevy n1, navstevy n2
        WHERE n1.P = n2.P
        AND n1.P = n.P
        AND n1.S <> n2.S
    )
    GROUP BY n.pijan
    HAVING COUNT(v.id) >= 2 
)
\end{verbatim}
\end{minipage}
TODO nie celkom dobre, prerobit
% \begin{align*}
%   \text{pcpmad} &\coloneqq ( \pi_{P} \circ \sigma_{COUNT(P) < 2} \circ
%   \Gamma_{P, COUNT(P)} ) ( \text{navstivil} \bowtie \text{vypil} )
%   \\ \text{pcnvru} &\coloneqq ( \pi_{P} \circ \sigma_{COUNT(\Sigma(C))
%     \neq 1} \circ \Gamma_{P, COUNT(\Sigma(C))} \circ \delta \circ
%   \pi_{P, \Sigma(C)} \circ \Gamma_{P, I, \Sigma(C)} ) (
%   \text{navstivil} \bowtie \text{vypil} \bowtie \text{capuje} )
%   \\ \text{answer} &\coloneqq (\delta \circ \pi_{K} ) (
%   \text{navstivil} \triangleright ( \text{pcpmad} \cup \text{pcnvru} )
%   )
% \end{align*}
\end{problem}

\begin{problem}
{\bf Zistite tržby v jednotlivých krčmách za predaj najmenej obľúbeného alkoholu (t.\,j. takého, že ho ľúbi najmenej pijanov; predpokladajte, že je len jeden). Vo výsledku uvádzajte len krčmy, ktoré ten alkohol čapujú.}\\[5mm]

TODO neobsahuje nuly
\begin{align*}
najmenejludi & = \Gamma_{C=MIN(C)}(\Gamma_{A,C = COUNT(\delta P)}(lubi))\\
nepopularnyA & = \pi_A(najmenejludi \bowtie (\Gamma_{A,C = COUNT(\delta P)}(lubi)))\\
nakupy & = capuje \bowtie (nepopularnyA \bowtie (navstivil \bowtie vypil)))\\
\answer & = \Gamma_{K,SUM(C*M)} (nakupy)
\end{align*}
\end{problem}

\begin{problem}
{\bf Zistite počet pijanov, ktorí už niečo pili a majú presne tie isté chute ako Ignác (t.\,j. ľúbia presne tie isté alkoholy). }\\[5mm]

\begin{align*}
{\rm alkoholy\_lubi\_Ignac} :=& \Pi_A(\sigma_{P = Ignac}(\lubi))\\
{\rm alkoholy\_nelubi\_Ignac} :=&  \delta(\Pi_A(\lubi\antijoin {\rm alkoholy\_lubi\_Ignac})\\
{\rm pijani\_lubia\_nieco\_naviac} :=& \delta(\Pi_P(\lubi\bowtie {\rm alkoholy\_nelubi\_Ignac})))\\
{\rm pijani\_rovnaky\_pocet\_ako\_Ignac} :=& \Pi_P(\Gamma_{P, C=COUNT(A)}(\lubi) \bowtie \Gamma_{C = COUNT(A)}({\rm alkoholy\_lubi\_Ignac}))\\
\answer :=& \Gamma_{C=COUNT(P)-1}(\delta(\Pi_P(\navstivil\bowtie \vypil))\bowtie\\
    & ({\rm pijani\_rovnaky\_pocet\_ako\_Ignac} - {\rm pijani\_lubia\_nieco\_naviac}))\\
\end{align*}

\def\ctt{\textbf{CREATE TEMPORARY TABLE}}
\def\select{\textbf{SELECT}}
\def\from{\textbf{FROM}}
\def\where{\textbf{WHERE}}
\def\as{\textbf{AS}}
\def\distinct{\textbf{DISTINCT}}
\def\join{\textbf{JOIN}}
\def\on{\textbf{ON}}
\def\exists{\textbf{EXISTS}}
\def\notexists{\textbf{NOT EXISTS}}
\def\groupby{\textbf{GROUP BY}}
\def\having{\textbf{HAVING}}

\ctt\ ignacChute \as\ (\\
\hsp\select\ \distinct A\\
\hsp\from\ lubi\\
\hsp\where\ P = `Ignac'\\
);

\ctt\ pijanicopili \as\ (\\
\hsp\select\ \distinct\ n.P\\
\hsp\from\ navstivil n \join\ vypil v \on\ n.I = v.I\\
);

\ctt\ alkoholyNavyse \as\ (\\
\hsp\select\ \distinct\ P\\
\hsp\from\ lubi\\
\hsp\where\ \notexists\ (\\
\hsp\hsp\select\ 1\\
\hsp\hsp\from\ ignacChute\\
\hsp\hsp\where\ ignacChute.A = lubi.A)\\
);

\ctt\ rovnakechute \as\ (\\
\hsp\select\ lubi.P\\
\hsp\from\ pijanicopili p\\
\hsp\hsp\join\ lubi \on\ p.P = lubi.P\\
\hsp\hsp\join\ ignacchute ig \on\ ig.A = lubi.A\\
\hsp\where\ \notexists\ (\\
\hsp\hsp\select\ 1 \from\ alkoholyNavyse an\\
\hsp\hsp\where\ an.P = lubi.P\\
\hsp\groupby\ lubi.P\\
\hsp\having\ COUNT(lubi.A) = (\select\ COUNT(*) \from\ ignacChute)\\
);

\select\ COUNT(*)\\
\from\ rovnakechute\\
\end{problem}

