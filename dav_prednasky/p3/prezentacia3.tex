\PassOptionsToPackage{dvipsnames}{xcolor}

\documentclass[12pt]{beamer}
\usetheme{default}
\usecolortheme{crane}

\usepackage[utf8x]{inputenc}
\usepackage[T1]{fontenc}
\usepackage[slovak]{babel}
\usepackage{ucs} % unicode

\usepackage{amsmath}
\usepackage{amsmath, amssymb}
\usepackage{hyperref, url}
\usepackage{graphicx}
\usepackage{array}
\usepackage{alltt}

%\setbeamersize{text margin left=1pt,text margin right=1pt}
\setbeamertemplate{footline}[frame number]
\beamertemplatenavigationsymbolsempty

% https://www.overleaf.com/learn/latex/Using_colours_in_LaTeX
\def\blue#1{\textcolor{Cerulean}{#1}}
\def\green#1{\textcolor{LimeGreen}{#1}}

% database-related stuff
\DeclareMathOperator{\join}{\bowtie}
\DeclareMathOperator{\antijoin}{\rhd}

\DeclareMathOperator{\lubi}{lubi}
\DeclareMathOperator{\capuje}{capuje}
\DeclareMathOperator{\navstivil}{navstivil}
\DeclareMathOperator{\vypil}{vypil}
\DeclareMathOperator{\answer}{answer}


\title{Agregácia v SQL}
\author{Ján Mazák}
\institute{FMFI UK Bratislava}
\date{}


\begin{document}

\frame{\titlepage}



\begin{frame}[fragile]{Agregácia}
Niekedy nás nezaujímajú jednotlivé záznamy relácie, ale jedna agregátna hodnota.
\begin{alltt}
/* počet zamestnancov */

SELECT \alert{COUNT}(e.emp_id)
FROM employee e
\end{alltt}
Agregačné funkcie: COUNT, SUM, AVG, MAX, MIN\dots
\end{frame}

\begin{frame}[fragile]{Agregácia}
Riadky možno rozdeliť do skupín pomocou \alert{GROUP BY}.\\
Na výstupe bude pre každú skupinu 1 riadok.
\begin{alltt}
/* počet zamestnancov v jednotlivých oddeleniach */

SELECT e.dept_id, \alert{COUNT}(e.emp_id) AS c
FROM employee e
\alert{GROUP BY} e.dept_id
\end{alltt}
\end{frame}

\begin{frame}[fragile]{Agregácia}
\blue{HAVING} umožňuje filtrovať skupiny.
(Pomenovanie atribútov vytvorených agregáciou za SELECT sa nedá použiť za \blue{HAVING}.)
{
\small
\begin{alltt}
/* počet zamestnancov v oddeleniach s > 1 zamestnancom */
\end{alltt}
}
\begin{alltt}
SELECT e.dept_id, \alert{COUNT}(e.emp_id) AS c
FROM employee e
\alert{GROUP BY} e.dept_id
\blue{HAVING} COUNT(e.emp_id) > 1
\end{alltt}
\bigskip
\end{frame}

\begin{frame}[fragile]{Agregácia}
Mimo dotazu môžu ísť len agregované hodnoty a atribúty, na ktorých sa všetky záznamy v skupine zhodujú
(toto je z pohľadu databázy zaručené len vtedy, ak sa ten atribút nachádza za GROUP BY).
\begin{alltt}
/* počet zamestnancov v jednotlivých oddeleniach */

SELECT \blue{d.name}, \alert{COUNT}(e.emp_id) AS c
FROM employee e
    JOIN department d ON e.dept_id = d.dept_id
\alert{GROUP BY} d.dept_id, \blue{d.name}
\end{alltt}
(Pridanie \verb|d.name| nemá vplyv na rozdelenie riadkov do skupín.)
\end{frame}

\begin{frame}[fragile]{Agregácia}
Do skupín možno deliť aj podľa viacerých atribútov súčasne (riadky v skupine sa musia zhodovať na všetkých) alebo podľa nejakej vypočítanej hodnoty.
\begin{alltt}
/* počet zamestnancov v skupinách
podľa platu zaokrúhleného na stovky
v jednotlivých oddeleniach */

SELECT d.name, ROUND(e.salary, -2) AS salary,
    \alert{COUNT}(e.emp_id) AS c
FROM employee e
    JOIN department d ON e.dept_id = d.dept_id
\alert{GROUP BY} d.dept_id, d.name, ROUND(e.salary, -2)
\end{alltt}
\end{frame}

\begin{frame}[fragile]{Agregácia}
Postupujte opatrne pri aplikovaní agregačných funkcií na NULL a na potenciálne prázdnu množinu záznamov.
Výsledky neraz nie sú intuitívne, treba podrobne naštudovať dokumentáciu a overiť správanie pre konkrétny DBMS.
Napríklad:
\begin{itemize}
\item \verb|COUNT(stĺpec)| ignoruje NULL, ale \verb|COUNT(*)| ich zaráta (aspoň v MySQL)
\item AVG pre neprázdnu množinu ignoruje NULL a výsledok tak môže byť veľmi nereprezentatívny; pre prázdnu vráti NULL
\end{itemize}
\end{frame}

\begin{frame}[fragile]{Agregácia --- GROUPING SETS}
Možnosť viac GROUP BY v jednom dotaze:
\begin{alltt}
SELECT region, product, SUM(sales)
FROM sales
GROUP BY \alert{GROUPING SETS} (
  (region),          --- Sales by region
  (product),         --- Sales by product
  (region, product), --- Sales by region and product
  ()                 --- Grand total
)
\end{alltt}
Vhodné na rýchlejšie generovanie reportov.\\
Špeciálne prípady CUBE a ROLLUP.
\end{frame}

\begin{frame}[fragile]{WITH --- Common Table Expressions (CTE)}
WITH vytvorí reláciu existujúcu len počas výpočtu dotazu.
\bigskip
\begin{alltt}
\alert{WITH} \blue{pijanPocetAlkoholov}(pijan, c) AS (
    SELECT pijan, COUNT(DISTINCT alkohol)
    FROM lubi
    GROUP BY pijan
)
SELECT MAX(ppa.c)
FROM \blue{pijanPocetAlkoholov} ppa
\end{alltt}
\bigskip
WITH sa často používa pri viackrokovom agregovaní.
V jednom dotaze možno za WITH vymenovať aj viacero relácií oddelených čiarkou.
\end{frame}

\begin{frame}[fragile]{Záznamy, kde sa dosahuje extrém (arg max)}
\begin{alltt}
\alert{WITH} \blue{pijanPocetAlkoholov}(pijan, c) AS (
    SELECT pijan, COUNT(DISTINCT alkohol)
    FROM lubi
    GROUP BY pijan
)
SELECT ppa.pijan
FROM \blue{pijanPocetAlkoholov} ppa
WHERE ppa.c = (
    SELECT MAX(ppa2.c)
    FROM \blue{pijanPocetAlkoholov} ppa2
)
\end{alltt}
\bigskip
\end{frame}

\begin{frame}[fragile]{Agregácia --- chyby}
Pri výpočte \alert{arg max} nemožno použiť ORDER BY a LIMIT, pretože LIMIT vedie k fixnému počtu záznamov vo výsledku
a dopredu nevieme, pre koľko záznamov sa maximum nadobúda.
\\[5mm]

Pomocou ORDER BY a LIMIT 1 možno nájsť samotné maximum,
nie je to však vhodný spôsob:
\begin{itemize}
\item úmysel takého dotazu je skrytý (ORDER BY a LIMIT sa používajú najmä na niečo iné, čiže akoby ste pomocou špecifického LIMIT 1 menili význam ORDER BY)
\item výpočet je neefektívny (treba uložiť celý medzivýsledok, možno aj na disk, ak je veľký; až tak možno aplikovať ORDER BY)
\end{itemize}
\end{frame}

\begin{frame}[fragile]{Agregácia --- chyby}
Vnorenie agregovaných funkcií je neprípustné:
\begin{alltt}
SELECT \alert{MAX(COUNT(...))} ...

/* správne: najprv COUNT,
   potom MAX v ďalšom SELECTe */
\end{alltt}
\vspace{10mm}
Agregačnú funkciu nemožno aplikovať priamo na reláciu:
\begin{alltt}
WHERE x = \alert{MAX(r)} ...

/* správne: x = (SELECT MAX(...) FROM r) */
\end{alltt}
\end{frame}

\begin{frame}[fragile]{Agregácia --- chyby}
Dvojitá agregácia sa ťažko číta, neraz nefunguje (kvôli duplicite hodnôt vznikajúcej pri karteziánskom súčine) a je zdrojom chýb:
\begin{alltt}
FROM r, s
...
HAVING COUNT(r.x) = COUNT(s.y)

/* lepšie rozdeliť do osobitných SELECTov
a použiť join alebo jednu z agregácií
vypočítať vopred */
\end{alltt}
\end{frame}


\begin{frame}[fragile]{Vnorené dotazy (subqueries)}
Podľa toho, či výsledok vnoreného dotazu závisí od riadka, pre ktorý sa vyhodnocuje WHERE, rozlišujeme:
\begin{itemize}
\item \alert{nekorelované} --- nezávisí
\item \alert{korelované} --- zavisí
\end{itemize}
Nekorelované dotazy stačí počítať raz, kým korelované musíme počítať pre každý riadok nanovo.
Navyše narúšajú optimalizáciu (prepis dotazu do výpočtovo výhodnejšej podoby),
pretože vo všeobecnosti je ťažké hľadať súvis medzi programom a podprogramom.
Korelované vnorené dotazy sú preto niekedy veľmi pomalé; vyhýbajte sa im.
\end{frame}

\begin{frame}[fragile]{Vnorené dotazy (subqueries)}
Vnorené dotazy využívajúce IN, NOT IN, ANY, ALL apod. možno prepísať pomocou joinu alebo \alert{antijoinu}.
Antijoin počíta tie riadky z ľavej tabuľky, ktoré sa nejoinujú so žiadnymi riadkami pravej tabuľky.
\bigskip

Podobne ako join ho možno počítať \emph{rýchlo} --- napr. cez triedenie alebo hashovanie.
Pre korelované vnorené dotazy takúto optimalizáciu DBMS zväčša spraviť nevie.
\end{frame}

\begin{frame}[fragile]{Vnorené dotazy --- scalar subquery}
\footnotesize
\emph{Zamestnanci zarábajúci viac ako priemer v ich oddelení.}\\[7mm]
\begin{minipage}{.4\pdfpagewidth}
Correlated subquery:
\footnotesize
\begin{alltt}
SELECT e.emp_id, e.salary
FROM employee e
WHERE \blue{e.salary >} (
  SELECT AVG(e2.salary)
  FROM employee e2
  WHERE e2.dept_id =
    e.dept_id
);



\end{alltt}
\end{minipage}
\hfill\vline\hfill
\begin{minipage}{.4\pdfpagewidth}
Join:
\footnotesize
\begin{alltt}
WITH dept_avg AS (
  SELECT dept_id,
    AVG(salary) AS avg_salary
  FROM employee
  GROUP BY dept_id
)
SELECT e.emp_id, e.salary
FROM employee e
  JOIN dept_avg da
    ON e.dept_id = da.dept_id
WHERE \blue{e.salary >} da.avg_salary;
\end{alltt}
\end{minipage}
\end{frame}

\begin{frame}[fragile]{Vnorené dotazy --- NOT IN}
\footnotesize
\emph{Zamestnanci, ktorých nadriadený nie je priamym podriadeným prezidenta.}\\[7mm]
\begin{minipage}{.4\pdfpagewidth}
\alert{Un}correlated subquery:
\footnotesize
\begin{alltt}
SELECT e.emp_id, e.superior
FROM employee e
WHERE \blue{e.superior NOT IN} (
  SELECT \blue{s.emp_id}
  FROM employee s ----superior
  JOIN employee p ----president
    ON s.superior = p.emp_id
  WHERE p.job = 'president'
);

\end{alltt}
\end{minipage}
\hfill\vline\hfill
\begin{minipage}{.4\pdfpagewidth}
Antijoin:
\footnotesize
\begin{alltt}
SELECT e.emp_id, e.superior
FROM employee e
WHERE \blue{NOT EXISTS} (
  SELECT 1
  FROM employee s
  JOIN employee p
    ON s.superior = p.emp_id
  WHERE p.job = 'president'
    \blue{AND s.emp_id = e.superior}
);
\end{alltt}
\end{minipage}
\end{frame}

\begin{frame}[fragile]{Vnorené dotazy --- ALL}
\footnotesize
\emph{Zamestnanci zarábajúci viac ako všetci ich podriadení.}\\[7mm]
\begin{minipage}{.4\pdfpagewidth}
Correlated subquery:
\footnotesize
\begin{alltt}
SELECT e.name
FROM employee e
WHERE e.salary > \blue{ALL} (
  SELECT s.salary
  FROM employee s
  WHERE s.superior = e.emp_id
);

\end{alltt}
\end{minipage}
\hfill\vline\hfill
\begin{minipage}{.4\pdfpagewidth}
Antijoin:
\footnotesize
\begin{alltt}
SELECT e.name
FROM employee e
WHERE NOT EXISTS (
  SELECT 1
  FROM employee s
  WHERE s.superior = e.emp_id
  AND e.salary <= s.salary
);
\end{alltt}
\end{minipage}
\end{frame}

\begin{frame}[fragile]{Vnorené dotazy --- ANY}
\footnotesize
\emph{Zamestnanci zarábajúci menej ako nejaký ich podriadený.}\\[7mm]
\begin{minipage}{.4\pdfpagewidth}
Correlated subquery:
\footnotesize
\begin{alltt}
SELECT e.name
FROM employee e
WHERE e.salary < \blue{ANY} (
  SELECT s.salary
  FROM employee s
  WHERE s.superior = e.emp_id
);
\end{alltt}
\end{minipage}
\hfill\vline\hfill
\begin{minipage}{.4\pdfpagewidth}
Join:
\footnotesize
\begin{alltt}
SELECT DISTINCT e.name
FROM employee e
  JOIN employee sub
    ON sub.superior = e.emp_id
WHERE e.salary < sub.salary;


\end{alltt}
\end{minipage}
\end{frame}

\begin{frame}{Vnorené dotazy}
Nevýhody correlated subquery:
\begin{itemize}
\item Niekedy horšia čitateľnosť (inokedy lepšia).
\item Ťažšie debugovanie (lebo korelovaný dotaz nie je sám osebe zmysluplný a je previazaný s kontextom).
\item Zložitosť sťažuje budúce udržiavanie/modifikáciu kódu.
\item Horšie využívanie existujúcich indexov (vytvorenie ad-hoc zoznamu pre IN, ALL atď. preruší prepojenie na existujúce tabuľky).
\item Prekážka pre optimalizáciu (ťažko sa hľadá rýchly spôsob výpočtu); čím viac dát, tým väčší problém.
\item Treba si strážiť prítomnosť NULL.
\end{itemize}
\end{frame}

\begin{frame}[fragile]{Vnorené dotazy za FROM --- fuj, radšej WITH}
\scriptsize
\begin{alltt}
SELECT outer_query.emp_id, outer_query.name, outer_query.total_salary
FROM (
  SELECT middle_query.emp_id,
    middle_query.salary + COALESCE(middle_query.bonus, 0) AS total_salary
  FROM (
    SELECT e.emp_id, e.name, e.salary,
      (SELECT SUM(b.amount)
       FROM (
         SELECT bonus.emp_id, bonus.amount
         FROM employee_bonus bonus
         WHERE bonus.amount > 1000
       ) AS filtered_bonus
       WHERE filtered_bonus.emp_id = e.emp_id
      ) AS bonus
    FROM (
      SELECT emp_id, name, salary
      FROM employee
      WHERE salary > 50000
    ) AS e
  ) AS middle_query
) AS outer_query
WHERE outer_query.total_salary > 60000;
\end{alltt}
\end{frame}


\begin{frame}{Literatúra}
\begin{itemize}
\item {\scriptsize\url{https://www.postgresqltutorial.com/postgresql-aggregate-functions/}}
\item {\scriptsize\url{https://www.postgresqltutorial.com/} (Sections 4, 5, 7)}
\item {\scriptsize\url{https://learnsql.com/blog/error-with-group-by/}}
\item {\scriptsize\url{https://www.postgresql.org/docs/current/functions-aggregate.html}}
\end{itemize}
\end{frame}


\begin{frame}[fragile]{Na zamyslenie}
\begin{itemize}
\item Prejdite si túto prezentáciu pred každým písaním dotazov, najmä ukážky chýb a vhodných postupov.
    Niektorí študenti tieto chyby opakujú celý rok, vrátane písomky, ktorú potom musia písať opakovane.
\item Aký je rozdiel medzi filtrovaním pomocou HAVING a WHERE?
\item Aký je rozdiel medzi \verb|COUNT(x)| a \verb|COUNT(DISTINCT x)|\\
    z hľadiska výsledku a rýchlosti výpočtu?
\item Ako by sme spočítali počet zamestnancov, ktorých mená začínajú jednotlivými písmenami abecedy?
\end{itemize}
\end{frame}


\begin{frame}{Úlohy: SQL}
Databáza: \emph{lubi}(Pijan, Alkohol), \emph{capuje}(Krcma, Alkohol, Cena),
\emph{navstivil}(Id, Pijan, Krcma), \emph{vypil}(Id, Alkohol, Mnozstvo)
\begin{itemize}
	\item počet čapovaných alkoholov
	\item priemerná cena piva
	\item najdrahší čapovaný alkohol (všetky, ak ich je viac)
    \item pijan, ktorý vypil najmenej druhov alkoholu
	\item tržby jednotlivých krčiem
    \item krčma s najväčšou celkovou tržbou
    \item priem. suma prepitá pri 1 návšteve pre jednotlivé krčmy
    \item koľko najviac alkoholov, ktoré nik neľúbi, je v jednej krčme v ponuke?
\end{itemize}
\end{frame}


\end{document}


