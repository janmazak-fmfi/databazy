\PassOptionsToPackage{dvipsnames}{xcolor}

\documentclass[12pt]{beamer}
\usetheme{default}
\usecolortheme{crane}

\usepackage[utf8x]{inputenc}
\usepackage[T1]{fontenc}
\usepackage[slovak]{babel}
\usepackage{ucs} % unicode

\usepackage{amsmath}
\usepackage{amsmath, amssymb}
\usepackage{hyperref, url}
\usepackage{graphicx}
\usepackage{array}
\usepackage{alltt}

%\setbeamersize{text margin left=1pt,text margin right=1pt}
\setbeamertemplate{footline}[frame number]
\beamertemplatenavigationsymbolsempty

% https://www.overleaf.com/learn/latex/Using_colours_in_LaTeX
\def\blue#1{\textcolor{Cerulean}{#1}}
\def\green#1{\textcolor{LimeGreen}{#1}}

% database-related stuff
\DeclareMathOperator{\join}{\bowtie}
\DeclareMathOperator{\antijoin}{\rhd}

\DeclareMathOperator{\lubi}{lubi}
\DeclareMathOperator{\capuje}{capuje}
\DeclareMathOperator{\navstivil}{navstivil}
\DeclareMathOperator{\vypil}{vypil}
\DeclareMathOperator{\answer}{answer}


\title{Agregácia v SQL}
\author{Ján Mazák}
\institute{FMFI UK Bratislava}
\date{}


\begin{document}

\frame{\titlepage}



\begin{frame}[fragile]{Agregácia}
Niekedy nás nezaujímajú jednotlivé záznamy relácie, ale jedna agregátna hodnota.
\begin{alltt}
/* počet zamestnancov */

SELECT \alert{COUNT}(e.empno)
FROM employee e
\end{alltt}
Agregačné funkcie: COUNT, SUM, AVG, MAX, MIN\dots
\end{frame}

\begin{frame}[fragile]{Agregácia}
Riadky možno rozdeliť do skupín pomocou \alert{GROUP BY}.\\
Na výstupe bude pre každú skupinu 1 riadok.
\begin{alltt}
/* počet zamestnancov v jednotlivých oddeleniach */

SELECT e.deptno, \alert{COUNT}(e.empno) AS c
FROM employee e
\alert{GROUP BY} e.deptno
\blue{HAVING} COUNT(e.empno) > 1
\end{alltt}
\bigskip
\blue{HAVING} umožňuje filtrovať skupiny.\\
(Pomenovanie atribútov vytvorených agregáciou za SELECT sa nedá použiť za \blue{HAVING}.)
\end{frame}

\begin{frame}[fragile]{Agregácia}
Mimo dotazu môžu ísť len agregované hodnoty a atribúty, na ktorých sa všetky záznamy v skupine zhodujú
(toto je z pohľadu databázy zaručené len vtedy, ak sa ten atribút nachádza za GROUP BY).
\begin{alltt}
/* počet zamestnancov v jednotlivých oddeleniach */

SELECT \blue{d.name}, \alert{COUNT}(e.empno) AS c
FROM employee e
    JOIN department d ON e.deptno = d.deptno
\alert{GROUP BY} d.deptno, \blue{d.name}
\end{alltt}
(Pridanie \verb|d.name| nemá vplyv na rozdelenie riadkov do skupín.)
\end{frame}

\begin{frame}[fragile]{Agregácia}
Do skupín možno deliť aj podľa viacerých atribútov súčasne (riadky v skupine sa musia zhodovať na všetkých).
\begin{alltt}
/* počet zamestnancov v skupinách podľa platu
v jednotlivých oddeleniach */

SELECT d.name, d.salary, \alert{COUNT}(e.empno) AS c
FROM employee e
    JOIN department d ON e.deptno = d.deptno
\alert{GROUP BY} d.deptno, d.name, e.salary
\end{alltt}
\end{frame}

\begin{frame}[fragile]{Agregácia}
Postupujte opatrne pri aplikovaní agregačných funkcií na NULL a na potenciálne prázdnu množinu záznamov.
Výsledky neraz nie sú intuitívne, treba podrobne naštudovať dokumentáciu a overiť správanie pre konkrétny DBMS.
Napríklad:
\begin{itemize}
\item \verb|COUNT(stĺpec)| ignoruje NULL, ale \verb|COUNT(*)| ich zaráta (aspoň v MySQL)
\item AVG pre neprázdnu množinu ignoruje NULL a výsledok tak môže byť veľmi nereprezentatívny; pre prázdnu vráti NULL
\end{itemize}
\end{frame}

\begin{frame}[fragile]{WITH --- Common Table Expressions (CTE)}
WITH vytvorí reláciu existujúcu len počas výpočtu dotazu.
\bigskip
\begin{alltt}
\alert{WITH} \blue{pijanPocetAlkoholov}(pijan, c) AS (
    SELECT pijan, COUNT(DISTINCT alkohol)
    FROM lubi
)
SELECT MAX(ppa.c)
FROM \blue{pijanPocetAlkoholov} ppa
\end{alltt}
\bigskip
WITH sa často používa pri viackrokovom agregovaní.
V jednom dotaze možno za WITH vymenovať aj viacero relácií oddelených čiarkou.
\end{frame}

\begin{frame}[fragile]{Záznamy, kde sa dosahuje extrém (arg max)}
\begin{alltt}
\alert{WITH} \blue{pijanPocetAlkoholov}(pijan, c) AS (
    SELECT pijan, COUNT(DISTINCT alkohol)
    FROM lubi
)
SELECT ppa.pijan
FROM \blue{pijanPocetAlkoholov} ppa
WHERE ppa.c = (
    SELECT MAX(ppa2.c)
    FROM \blue{pijanPocetAlkoholov} ppa2
)
\end{alltt}
\bigskip
\end{frame}


\begin{frame}{Literatúra}
\begin{itemize}
\item {\scriptsize\url{https://www.postgresqltutorial.com/postgresql-aggregate-functions/}}
\item {\scriptsize\url{https://www.postgresqltutorial.com/} (Sections 4, 5, 7)}
\item {\scriptsize\url{https://learnsql.com/blog/error-with-group-by/}}
\item {\scriptsize\url{https://www.postgresql.org/docs/current/functions-aggregate.html}}
\end{itemize}
\end{frame}


\begin{frame}{Úlohy: SQL}
Databáza: \emph{lubi}(Pijan, Alkohol), \emph{capuje}(Krcma, Alkohol, Cena),
\emph{navstivil}(Id, Pijan, Krcma), \emph{vypil}(Id, Alkohol, Mnozstvo)
\begin{itemize}
	\item počet čapovaných alkoholov
	\item priemerná cena piva
	\item najdrahší čapovaný alkohol (všetky, ak ich je viac)
    \item pijan, ktorý vypil najmenej druhov alkoholu
	\item tržby jednotlivých krčiem
    \item krčma s najväčšou celkovou tržbou
    \item priem. suma prepitá pri 1 návšteve pre jednotlivé krčmy
    \item koľko najviac alkoholov, ktoré nik neľúbi, je v jednej krčme v ponuke?
\end{itemize}
\end{frame}


\end{document}


