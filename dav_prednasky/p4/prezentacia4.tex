\PassOptionsToPackage{dvipsnames}{xcolor}

\documentclass[12pt]{beamer}
\usetheme{default}
\usecolortheme{crane}
%\usetheme{Madrid}

\usepackage[utf8x]{inputenc}
\usepackage[T1]{fontenc}
\usepackage[slovak]{babel}
\usepackage{ucs}
\usepackage{amsmath}
\usepackage{graphicx}
\usepackage{array}
\usepackage{amsmath, amssymb}
\usepackage{hyperref, url}
\usepackage{alltt}
%\usepackage[inline]{asymptote}

%\setbeamersize{text margin left=1pt,text margin right=1pt}
\setbeamertemplate{footline}[frame number]
\beamertemplatenavigationsymbolsempty

\let\o=\vee
\let\a=\wedge
\let\bigo=\bigvee
\let\biga=\bigwedge

% https://www.overleaf.com/learn/latex/Using_colours_in_LaTeX
\def\blue#1{\textcolor{Cerulean}{#1}}

\title{Relačná algebra}
\author{Ján Mazák}
\institute{FMFI UK Bratislava}
\date{}
%\date % (optional)
%{23. 9. 2019}

% database-related stuff
\DeclareMathOperator{\join}{\bowtie}
\DeclareMathOperator{\antijoin}{\rhd}

\DeclareMathOperator{\COUNT}{\textrm{COUNT}}
\DeclareMathOperator{\SUM}{\textrm{SUM}}
\DeclareMathOperator{\MAX}{\textrm{MAX}}


\DeclareMathOperator{\osoba}{osoba}
\DeclareMathOperator{\firma}{firma}
\DeclareMathOperator{\vlastni}{vlastni}
\DeclareMathOperator{\ponuka}{ponuka}
\DeclareMathOperator{\chce}{chce}
\DeclareMathOperator{\lubi}{lubi}
\DeclareMathOperator{\capuje}{capuje}
\DeclareMathOperator{\navstivil}{navstivil}
\DeclareMathOperator{\vypil}{vypil}
\DeclareMathOperator{\answer}{answer}


\begin{document}

\frame{\titlepage}

\begin{frame}[fragile]{Common Table Expressions (CTE)}
WITH vytvorí reláciu existujúcu len počas výpočtu dotazu.
\bigskip
\begin{alltt}
\alert{WITH} pijanPocetA(pijan, c) AS (
    SELECT pijan, COUNT(DISTINCT alkohol)
    FROM lubi
)
SELECT MAX(ppa.c)
FROM pijanPocetA ppa
\end{alltt}
\end{frame}

\begin{frame}[fragile]{Relačná algebra}
\begin{itemize}
    \item interný jazyk, do ktorého sa prekladajú všetky dotazy
    \item tiež jazyk na formalizáciu relačného modelu a matematické dokazovanie
    \item zachytáva postup výpočtu dotazu pomocou\\ \alert{logických operátorov} (nezohľadňujú fyzické uloženie dát)
    \item vstupom aj výstupom operátora je relácia
    \item k danému dotazu možno zostrojiť rôzne zápisy (operátorové stromy) v relačnej algebre, databáza si sama vyberie ten, čo pokladá za najvhodnejší
\end{itemize}
\end{frame}

\begin{frame}[fragile]{Relačná algebra}
\includegraphics[scale=.8]{query1x.jpg}
\includegraphics[scale=.8]{query2x.jpg}\\[3mm]
\tiny{\url{https://dbis-uibk.github.io/relax/calc/gist/379b0fdd72490e8e634bb193f109d4a8}}
\end{frame}

\begin{frame}[fragile]{Logické operátory}
\begin{itemize}
    \item $\pi$ --- projekcia (vyberáme stĺpce)
    \item $\sigma$ --- selekcia (vyberáme riadky)
    \item $\rho$ --- premenovanie (relácie či atribútu)
    \item $\times$ --- karteziánsky súčin
    \item $\bowtie$ --- theta-join (natural join / join s podmienkami)
    \item $\antijoin$ --- antijoin (riadky 1. relácie, ktoré sa nedajú joinovať\\ \hskip 2.5cm so žiadnymi riadkami 2. relácie)
    \item $-, \cup, \cap$ --- rozdiel, zjednotenie, prienik množín
    \item $\Gamma$ or $\gamma$ --- group by
\end{itemize}
\end{frame}

\begin{frame}[fragile]{Logické operátory}
\begin{alltt}
SELECT a1, a2, COUNT(a3) AS b
FROM r1, r2
WHERE c1 OR c2
GROUP BY g1, g2
HAVING h1 AND h2
\end{alltt}
$$\pi_{a1, a2, b}(\sigma_{h1\a h2}(\Gamma_{g1, g2, COUNT(a3)\rightarrow b} (r1\bowtie_{c1 \o c2} r2)))$$
\bigskip
$$j := r_1\bowtie_{c_1 \o c_2} r_2$$
$$\pi_{a_1, a_2, b}(\sigma_{h_1\a h_2}(\Gamma_{g_1, g_2, COUNT(a_3)\rightarrow b} (j)))$$
\end{frame}

\begin{frame}{Literatúra}
\begin{itemize}
\item {\scriptsize\url{https://www.db-book.com/slides-dir/PDF-dir/ch2.pdf}}
\item {\scriptsize\url{https://dbis-uibk.github.io/relax/calc/gist/379b0fdd72490e8e634bb193f109d4a8}}
\end{itemize}
\end{frame}

\begin{frame}{Úlohy: relačná algebra}
Databáza: \emph{lubi}(Pijan, Alkohol), \emph{capuje}(Krcma, Alkohol, Cena),
\emph{navstivil}(Id, Pijan, Krcma), \emph{vypil}(Id, Alkohol, Mnozstvo)
\begin{itemize}
	\item pijani, čo ľúbia pivo
	\item koľko stojí najlacnejšie pivo?
    \item alkoholy, ktoré čapujú, ale nik ich neľúbi
    \item alkoholy, ktoré čapujú, ale nik ich nepil
	\item najdrahší čapovaný alkohol (všetky, ak ich je viac)
	\item pijani, ktorí navštívili všetky krčmy, čo niečo čapujú
    \item krčma s najväčšou celkovou tržbou
\end{itemize}
\end{frame}


\end{document}


