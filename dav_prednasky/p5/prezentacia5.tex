\PassOptionsToPackage{dvipsnames}{xcolor}

\documentclass[12pt]{beamer}
\usetheme{default}
\usecolortheme{crane}

\usepackage[utf8x]{inputenc}
\usepackage[T1]{fontenc}
\usepackage[slovak]{babel}
\usepackage{ucs} % unicode

\usepackage{amsmath}
\usepackage{amsmath, amssymb}
\usepackage{hyperref, url}
\usepackage{graphicx}
\usepackage{array}
\usepackage{alltt}

%\setbeamersize{text margin left=1pt,text margin right=1pt}
\setbeamertemplate{footline}[frame number]
\beamertemplatenavigationsymbolsempty

% https://www.overleaf.com/learn/latex/Using_colours_in_LaTeX
\def\blue#1{\textcolor{Cerulean}{#1}}
\def\green#1{\textcolor{LimeGreen}{#1}}

% database-related stuff
\DeclareMathOperator{\join}{\bowtie}
\DeclareMathOperator{\antijoin}{\rhd}

\DeclareMathOperator{\lubi}{lubi}
\DeclareMathOperator{\capuje}{capuje}
\DeclareMathOperator{\navstivil}{navstivil}
\DeclareMathOperator{\vypil}{vypil}
\DeclareMathOperator{\answer}{answer}

\title{Rekurzia a window functions}
\author{Ján Mazák}
\institute{FMFI UK Bratislava}
\date{}


\begin{document}

\frame{\titlepage}

\begin{frame}[fragile]{Window functions}
\begin{itemize}
\item Použitie GROUP BY vedie k nahradeniu celej skupiny jediným riadkom.
\item Niekedy však chceme zachovať pôvodné riadky a len k nim doplniť hodnotu súvisiacu s inými riadkami v skupine.
\item Riešenie: \alert{window functions}. Tieto funkcie vidia nielen jeden riadok, ale aj \uv{okolité} riadky
    (presná definícia je súčasťou dotazu v časti \alert{OVER}).
\item Window functions nevidia riadky odfiltrované vo WHERE a HAVING.
\item Ak chceme filtrovať výsledok podľa vypočítanej hodnoty window funkcie, treba použiť vnorený dotaz.
\end{itemize}
\end{frame}

\begin{frame}[fragile]{Window functions}
\begin{alltt}
SELECT empno, name, salary,
       \green{avg}(salary) \alert{OVER (PARTITION BY deptno)} avg_s
FROM employee;
\end{alltt}
\begin{alltt}
\blue{
 empno |  name  | salary  |          avg_s
-------------+----------------+------------------+-------------------------------------------
    11 | King   | 5000.00 | 2600.0000000000000000
    13 | Clark  | 1500.00 | 2600.0000000000000000
    24 | Miller | 1300.00 | 2600.0000000000000000
    14 | Jones  | 2975.00 | 2175.0000000000000000
    20 | Ford   | 3000.00 | 2175.0000000000000000
    21 | Smith  |  800.00 | 2175.0000000000000000
    ...
}
\end{alltt}
\end{frame}

\begin{frame}[fragile]{Window functions}
\begin{alltt}
SELECT deptno, name, salary,
       \green{rank()} \alert{OVER (PARTITION BY deptno
                    ORDER BY salary DESC)}
FROM employee;
\end{alltt}
\begin{alltt}
\blue{
 deptno |  name  | salary  | rank
 -------------+----------------+------------------+-------------------------------------------
     10 | King   | 5000.00 |    1
     10 | Clark  | 1500.00 |    2
     10 | Miller | 1300.00 |    3
     20 | Ford   | 3000.00 |    1
     20 | Scott  | 3000.00 |    1
     20 | Jones  | 2975.00 |    3
     20 | Adams  | 1100.00 |    4
    ...
}
\end{alltt}
\end{frame}

\begin{frame}[fragile]{Window functions}
Funkcie, ktoré možno použiť ako window functions:
\begin{itemize}
\item bežné agregačné funkcie\\
    (môžu sa však správať trochu odlišne!)
\item rank(), percent\_rank()
\item nth\_value() --- element in the n-th row of the window
\item \dots
\end{itemize}
\end{frame}

\begin{frame}[fragile]{Window functions}
\begin{itemize}
\item Ak vynecháme PARTITION BY, partícia obsahuje všetky riadky výstupu.
\begin{alltt}
    SELECT salary, sum(salary) \alert{OVER ()}
    FROM empsalary;
\end{alltt}
\vspace*{4mm}
\pause
\item Okno možno pomenovať a použiť viackrát.
\begin{alltt}
    SELECT
        sum(salary) OVER \blue{w},
        avg(salary) OVER \blue{w}
    FROM employee
    \alert{WINDOW \blue{w} AS (PARTITION BY deptno
                ORDER BY salary DESC);
}
\end{alltt}
\end{itemize}
\end{frame}

\begin{frame}[fragile]{Window frame}
\begin{itemize}
\item Každý riadok má svoje vlastné okno (window frame) --- zoznam riadkov, nad ktorými sa vykoná window function.
\item Okno závisí od použitej funkcie. Default window frame možno v rámci dotazu modifikovať.
\item Komplikovaná syntax. Nasledujúce dve formulácie sú ekvivalentné významovo, ale vraj nie z hľadiska efektívnosti výpočtu\dots
\begin{alltt}
\blue{ROWS UNBOUNDED PRECEDING EXCLUDE CURRENT ROW}
\green{ROWS BETWEEN UNBOUNDED PRECEDING AND 1 PRECEDING}
\end{alltt}
\end{itemize}
\end{frame}

\begin{frame}[fragile]{Window frame}
\begin{alltt}
SELECT salary, \green{sum}(salary) \alert{OVER (ORDER BY salary)}
FROM employee;
\end{alltt}
{\scriptsize
\begin{alltt}
\blue{
 salary  |   sum
------------------+-----------------
  800.00 |   800.00
  950.00 |  1750.00
 1100.00 |  2850.00
 1300.00 |  4150.00
 1500.00 |  7150.00   \green{<---- pribudlo 3000, nie 1500}
 1500.00 |  7150.00
 1750.00 |  8900.00
}
\end{alltt}
}
Okno: riadky v partícii po daný riadok plus riadky s rovnakou hodnotou.
\end{frame}

\begin{frame}[fragile]{Array datatype}
Databázy bežne podporujú polia. Dátový typ v PostgreSQL:
\begin{alltt}
  CREATE TABLE x (
      y  integer[],
      z  text[3][3]
  );
\end{alltt}
Verzia konformná s SQL štandardom:
\begin{alltt}
  CREATE TABLE x (
      y  integer ARRAY[4],
      z  text ARRAY[][]
  );
\end{alltt}
V súčasnej implementácii je deklarovaná veľkosť poľa ignorovaná.
\end{frame}

\begin{frame}[fragile]{Arrays}
Pole možno vytvoriť priamo v rámci dotazu.
\begin{alltt}
SELECT ARRAY[1,2] || ARRAY[3,4] AS a;
SELECT ARRAY[1, 2] || '\{3, 4\}';
\end{alltt}
Na prácu s poľom existuje množstvo funkcií, napr. \verb|array_prepend|, \verb|array_length|.
\begin{alltt}
SELECT array_position(
    ARRAY['mon','tue','wed','thu','fri','sat','sun'],
    'mon'
  );
SELECT * FROM x WHERE 47 = ANY (some_array);
SELECT * FROM x WHERE 47 = ALL (some_array);
\end{alltt}
\end{frame}

\begin{frame}[fragile]{Arrays}
Pole tiež možno získať aplikovaním agregačnej funkcie.
\begin{alltt}
SELECT e.name, \alert{array_agg}(p.name)
FROM employee e JOIN project p ON p.empno = e.empno
GROUP BY e.empno, e.name;
\end{alltt}
\begin{alltt}
\blue{
  name  |     array_agg
---------------+-----------------------------------
 Jones  | \{Enviro2,Nuclear1\}
 Smith  | \{Enviro1\}
 Turner | \{Nuclear1\}
 Blake  | \{Enviro1\}
}
\end{alltt}
\end{frame}


\begin{frame}[fragile]{Rekurzia: cez CTE (Common Table Expressions)}
\alert{WITH RECURSIVE} umožňuje rekurzívne sa odvolať na reláciu.
\bigskip
\begin{alltt}
\alert{WITH RECURSIVE} cte_name AS(
    SELECT ...  ---- non-recursive term
  UNION [ALL]
    SELECT ...  ---- recursive term
)
SELECT * FROM cte_name;
\end{alltt}
\end{frame}

\begin{frame}[fragile]{Rekurzia}
%\begin{minipage}{.4\pdfpagewidth}
\begin{alltt}
\alert{WITH RECURSIVE} t(n) AS (
    SELECT 1
  UNION ALL
    SELECT n+1 FROM t \green{WHERE n < 5}
)
SELECT n FROM t;
\end{alltt}
%\end{minipage}
\end{frame}

\begin{frame}[fragile]{Výpočet rekurzie --- iterácia (seminaive evaluation)}
\begin{enumerate}
\item Evaluate the \blue{non-recursive} term. For UNION (but not UNION ALL), discard duplicate rows.
Include all remaining rows in the result of the recursive query, and also place them in a temporary working table.

\item Evaluate the \blue{recursive} term, substituting the current contents of the working table for the recursive self-reference.
For UNION, discard duplicate rows and rows that duplicate any previous result row.
Include all remaining rows in the result of the recursive query, and also place them in a temporary intermediate table.

\item If the intermediate table is empty, stop. Otherwise replace the working table with the intermediate table and go to 2.
\end{enumerate}
\end{frame}

\begin{frame}[fragile]{Rekurzia}
Rekurzívny odkaz možno použiť len raz a pre jednu reláciu (PostgreSQL).
\alert{Nekorektné} využitie:
%\begin{minipage}{.4\pdfpagewidth}
\begin{alltt}
WITH RECURSIVE t(n) AS (
    SELECT 1
  UNION ALL
    (SELECT n+1 FROM \blue{t} WHERE n < 5
        UNION ALL
     SELECT n-1 FROM \blue{t} WHERE n < 5)
)
SELECT n FROM t;
\end{alltt}
%\end{minipage}
\end{frame}

\begin{frame}[fragile]{Rekurzia}
Rekurzívny výpočet nemusí skončiť, zodpovednosť je na autorovi dotazu.
\bigskip

\begin{alltt}
WITH RECURSIVE t(n) AS (
    SELECT 1
  UNION ALL
    SELECT n+1 FROM t
)
SELECT n FROM t \green{WHERE n < 5};
\end{alltt}
\end{frame}

\begin{frame}[fragile]{Rekurzia}
Rekurzia zvyšuje vyjadrovaciu silu jazyka SQL.
\begin{itemize}
\item Rekurziou sa dá vyjadriť \blue{tranzitívny uzáver},\\ t.\,j. existencia cesty neobmedzenej dĺžky.
\item Bez rekurzie možno vyjadriť len cestu fixnej dĺžky (pomocou opakovaného joinu).
\end{itemize}
\end{frame}

\begin{frame}[fragile]{Prehľadávanie grafov}
Pomocou rekurzie možno zapísať prehľadávanie stromu.\\
\phantom{x}\\ % align with the next slide
\phantom{x}\\
\bigskip

\begin{alltt}
WITH RECURSIVE search_tree(vFrom, vTo) AS (
    SELECT t.vFrom, t.vTo
    FROM tree t
  UNION ALL
    SELECT st.vFrom, t.vTo
    FROM search_tree st, tree t
    WHERE t.vFrom = st.vTo
)
SELECT * FROM search_tree;
\end{alltt}
\end{frame}

\begin{frame}[fragile]{Prehľadávanie grafov}
Poradie navštívených vrcholov závisí od implementácie rekurzie,
ale vhodným usporiadaním vieme dosiahnuť \alert{prehľadávanie do šírky}.
\bigskip
\begin{alltt}
WITH RECURSIVE search_tree(vFrom, vTo, \alert{depth}) AS (
    SELECT t.vFrom, t.vTo, \alert{1}
    FROM tree t
  UNION ALL
    SELECT st.vFrom, t.vTo, \alert{depth + 1}
    FROM search_tree st, tree t
    WHERE t.vFrom = st.vTo
)
SELECT * FROM search_tree \alert{ORDER BY depth};
\end{alltt}
\end{frame}

\begin{frame}[fragile]{Prehľadávanie grafov}
Poradie navštívených vrcholov závisí od implementácie rekurzie,
ale vhodným usporiadaním vieme dosiahnuť \alert{prehľadávanie do hĺbky}.
\bigskip
\begin{alltt}
WITH RECURSIVE search_tree(vFrom, vTo, \alert{path}) AS (
    SELECT t.vFrom, t.vTo, \alert{ARRAY[t.vFrom, t.vTo]}
    FROM tree t
  UNION ALL
    SELECT st.vFrom, t.vTo, \alert{path || t.vTo}
    FROM search_tree st, tree t
    WHERE t.vFrom = st.vTo
)
SELECT * FROM search_tree \alert{ORDER BY path};
\end{alltt}
\end{frame}

\begin{frame}[fragile]{Prehľadávanie grafov}
Ak chceme ustrážiť koniec rekurzie, treba sa vyhýbať \alert{zacykleniu}.
\bigskip
{\small
\begin{alltt}
WITH RECURSIVE
    search_graph(vFrom, vTo, depth, \green{path}, \blue{is_cycle})
AS (
    SELECT g.vFrom, g.vTo, 1,
        \green{ARRAY[g.vFrom, g.vTo]}, \blue{false}
    FROM graph g
  UNION ALL
    SELECT sg.vFrom, g.vTo, sg.depth + 1,
        \green{path || g.vTo}, \blue{g.vTo = ANY(path)}
    FROM search_graph sg, graph g
    WHERE g.vFrom = sg.vTo AND \blue{NOT is_cycle}
)
SELECT * FROM search_graph ORDER BY is_cycle, path;
\end{alltt}
}
\end{frame}


\begin{frame}{Literatúra}
\begin{itemize}
\item {\scriptsize\url{https://www.sqltutorial.org/sql-window-functions/}}
\item {\scriptsize\url{https://www.sqlite.org/windowfunctions.html}}
\item {\scriptsize\url{https://www.postgresql.org/docs/current/tutorial-window.html}}
\item {\scriptsize\url{https://www.postgresql.org/docs/current/arrays.html}}
\item {\scriptsize\url{https://www.postgresql.org/docs/current/queries-with.html}}
\item {\scriptsize\url{https://www.sqlite.org/lang_with.html}}
\end{itemize}
\end{frame}

\begin{frame}[fragile]{Úlohy}
Čo počíta tento dotaz?
\bigskip
{\small
\begin{alltt}
WITH RECURSIVE t(n) AS (
    SELECT 'foo'
  UNION ALL
    SELECT n || ' bar' FROM t WHERE length(n) < 20
)
SELECT n FROM t ORDER BY n;
\end{alltt}
}
\bigskip
\pause
Generuje reťazce v tvare \verb|foo( bar)|$^*$ postupne od najkratších.
\end{frame}


\begin{frame}[fragile]{Úlohy}
Súčet čísel od 1 do 100
\pause
\bigskip
{\small
\begin{alltt}
WITH RECURSIVE t(n) AS (
    VALUES (1)
  UNION ALL
    SELECT n+1 FROM t WHERE n < 100
)
SELECT sum(n) FROM t;
\end{alltt}
}
\end{frame}


\begin{frame}[fragile]{Úlohy}
Čo počíta tento dotaz?\\
Zmení sa to, ak UNION nahradíme UNION ALL?
\bigskip
{\small
\begin{alltt}
WITH RECURSIVE t(n) AS (
    SELECT 1
  \blue{UNION}
    SELECT 10-n FROM t
)
SELECT * FROM t ORDER BY n;
\end{alltt}
}
\end{frame}


\begin{frame}[fragile]{Úlohy}
Skončí výpočet tohto dotazu?
\bigskip
{\small
\begin{alltt}
WITH RECURSIVE t(n) AS (
    VALUES (1)
  UNION ALL
    SELECT n+1 FROM t
)
SELECT * FROM t LIMIT 10;
\end{alltt}
}
\bigskip
\pause
V PostgreSQL áno, ale vo všeobecnosti to závisí od implementácie LIMIT a rekurzie.
\end{frame}



\end{document}
