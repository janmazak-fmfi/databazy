\PassOptionsToPackage{dvipsnames}{xcolor}

\documentclass[12pt]{beamer}
\usetheme{default}
\usecolortheme{crane}
%\usetheme{Madrid}

\usepackage[utf8x]{inputenc}
\usepackage[T1]{fontenc}
\usepackage[slovak]{babel}
\usepackage{ucs}
\usepackage{amsmath}
\usepackage{graphicx}
\usepackage{array}
\usepackage{amsmath, amssymb}
\usepackage{hyperref, url}
\usepackage{alltt}
%\usepackage[inline]{asymptote}

%\setbeamersize{text margin left=1pt,text margin right=1pt}
\setbeamertemplate{footline}[frame number]
\beamertemplatenavigationsymbolsempty

\let\o=\vee
\let\a=\wedge
\let\bigo=\bigvee
\let\biga=\bigwedge

% https://www.overleaf.com/learn/latex/Using_colours_in_LaTeX
\def\blue#1{\textcolor{Cerulean}{#1}}

\title{Relačná algebra}
\author{Ján Mazák}
\institute{FMFI UK Bratislava}
\date{}
%\date % (optional)
%{23. 9. 2019}

% database-related stuff
\DeclareMathOperator{\join}{\bowtie}
\DeclareMathOperator{\antijoin}{\rhd}

\DeclareMathOperator{\COUNT}{\textrm{COUNT}}
\DeclareMathOperator{\SUM}{\textrm{SUM}}
\DeclareMathOperator{\MAX}{\textrm{MAX}}


\DeclareMathOperator{\osoba}{osoba}
\DeclareMathOperator{\firma}{firma}
\DeclareMathOperator{\vlastni}{vlastni}
\DeclareMathOperator{\ponuka}{ponuka}
\DeclareMathOperator{\chce}{chce}
\DeclareMathOperator{\lubi}{lubi}
\DeclareMathOperator{\capuje}{capuje}
\DeclareMathOperator{\navstivil}{navstivil}
\DeclareMathOperator{\vypil}{vypil}
\DeclareMathOperator{\answer}{answer}


\begin{document}

\frame{\titlepage}

\begin{frame}[fragile]{Common Table Expressions (CTE)}
WITH vytvorí reláciu existujúcu len počas výpočtu dotazu.
\bigskip
\begin{alltt}
\alert{WITH} pijanPocetA(pijan, c) AS (
    SELECT pijan, COUNT(DISTINCT alkohol)
    FROM lubi
)
SELECT MAX(ppa.c)
FROM pijanPocetA ppa
\end{alltt}
\end{frame}

\begin{frame}[fragile]{Common Table Expressions (CTE) --- rekurzia}
WITH RECURSIVE umožňuje použiť rekurziu.
\bigskip
\begin{alltt}
\alert{WITH RECURSIVE} cte_name AS(
    SELECT ...  -- non-recursive term
  UNION [ALL]
    SELECT ...  -- recursive term
)
SELECT * FROM cte_name;
\end{alltt}
\end{frame}

\begin{frame}[fragile]{Rekurzia}
%\begin{minipage}{.4\pdfpagewidth}
\begin{alltt}
\alert{WITH RECURSIVE} t(n) AS (
    SELECT 1
  UNION ALL
    SELECT n+1 FROM t WHERE n < 5
)
SELECT n FROM t;
\end{alltt}
%\end{minipage}
\end{frame}

\begin{frame}[fragile]{Výpočet rekurzie --- iterácia (seminaive evaluation)}
\begin{enumerate}
\item Evaluate the \blue{non-recursive} term. For UNION (but not UNION ALL), discard duplicate rows.
Include all remaining rows in the result of the recursive query, and also place them in a temporary working table.

\item Evaluate the \blue{recursive} term, substituting the current contents of the working table for the recursive self-reference.
For UNION, discard duplicate rows and rows that duplicate any previous result row.
Include all remaining rows in the result of the recursive query, and also place them in a temporary intermediate table.

\item If the intermediate table is empty, stop. Otherwise replace the working table with the intermediate table and go to 2.
\end{enumerate}
\end{frame}

\begin{frame}[fragile]{Rekurzia}
Rekurzívny odkaz možno použiť len raz a pre jednu reláciu (PostgreSQL).
\alert{Nekorektné} využitie:
%\begin{minipage}{.4\pdfpagewidth}
\begin{alltt}
WITH RECURSIVE t(n) AS (
    SELECT 1
  UNION ALL
    (SELECT n+1 FROM \blue{t} WHERE n < 5
        UNION ALL
     SELECT n-1 FROM \blue{t} WHERE n < 5)
)
SELECT n FROM t;
\end{alltt}
%\end{minipage}
\end{frame}

\begin{frame}[fragile]{Rekurzia}
Rekurzívny výpočet nemusí skončiť, zodpovednosť je na autorovi dotazu.
\bigskip

\begin{alltt}
WITH RECURSIVE t(n) AS (
    SELECT 1
  UNION ALL
    SELECT n+1 FROM t
)
SELECT n FROM t WHERE n < 5;
\end{alltt}
\end{frame}

\begin{frame}[fragile]{Rekurzia}
Rekurzia zvyšuje vyjadrovaciu silu jazyka SQL.
\begin{itemize}
\item Rekurziou sa dá vyjadriť \blue{tranzitívny uzáver},\\ t.\,j. existencia cesty neobmedzenej dĺžky.
\item Bez rekurzie možno vyjadriť len cestu fixnej dĺžky (pomocou opakovaného joinu).
\end{itemize}
\end{frame}


\begin{frame}{Literatúra}
\begin{itemize}
\item {\scriptsize\url{https://www.postgresql.org/docs/current/queries-with.html}}
\item {\scriptsize\url{https://www.sqlite.org/lang_with.html}}
\item {\scriptsize\url{https://www.postgresqltutorial.com/postgresql-tutorial/postgresql-recursive-query/}}
\end{itemize}
\end{frame}


\begin{frame}{Úlohy: SQL}
Databáza: \emph{osoba}(A), \emph{pozna}(Kto, Koho)
\begin{itemize}
	\item všetky osoby
    \item osoby, ktoré poznajú sysľa
    \item osoby, ktoré poznajú aspoň dve entity\\ (nemusia to byť osoby)
    \item osoby, ktoré nepoznajú nič a nikoho
    \item osoby, ktoré nepoznajú žiadne iné osoby
    \item osoby, ktoré poznajú iba Jožka
    \item osoby, ktoré pozná presne jedna osoba
%    \item osoby, ktoré poznajú všetkých známych svojich známych
%    \item osoby, ktoré majú všetky vzťahy symetrické
\end{itemize}
\end{frame}


\end{document}


